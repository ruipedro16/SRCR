\documentclass[a4paper, 11pt]{article}

\usepackage[utf8]{inputenc}
\usepackage[portuguese]{babel}
\usepackage{indentfirst}
\usepackage[pdfborder={0 0 0}]{hyperref}
\usepackage{a4wide}
\usepackage{graphicx}
\usepackage{float}
\usepackage{fancyhdr}
\usepackage{lastpage}
\usepackage{relsize}
\usepackage{listings}
\usepackage{chngcntr}
\usepackage{nomencl}

\lstset{
    basicstyle=\small,
    breaklines=true,
    frame=tb, 
    mathescape=true,
    showstringspaces=false,
    captionpos=b,
    keepspaces=true,
    escapeinside=``,
    language=Prolog
}

\title{Sistemas de Representação de \\ Conhecimento e Raciocínio\\ [0.8em] 
\smaller{}Programação em Lógica Estendida e Conhecimento Imperfeito}
\author{André Ferreira (a64296) \and Maria João Moreira (a89540) \and Rúben Rodrigues (a80960) 
\and Rui Fernandes (a89138)
\and Rui Morais (a76650)}
\date{Maio 2021}

\renewcommand\labelitemi{---}
\renewcommand{\nomname}{Lista de Siglas e Acrónimos}
\counterwithin{figure}{section}

\makenomenclature

\begin{document}

\begin{titlepage}
    \begin{center}
        \begin{minipage}{0.75\linewidth}
            \centering
            \includegraphics[width=0.4\textwidth]{img/EEUM.png}\par\vspace{1cm}
            \vspace{1.5cm}
            \href{https://www.uminho.pt/PT}{\scshape\LARGE Universidade do Minho} \par
            \vspace{1cm}
            \href{https://www.di.uminho.pt/}{\scshape\Large Departamento de Informática} \par
            \vspace{1.5cm}
            \maketitle
        \end{minipage}
    \end{center}
    \vspace{2cm}
    \thispagestyle{empty}
    \clearpage
\end{titlepage}

\pagenumbering{roman}

\begin{abstract}

O presente relatório tem como objetivo descrever a solução criada pelo grupo para a 
problemática proposta pelos docentes
da Unidade Curricular de 
\href{https://miei.di.uminho.pt/plano_estudos.html#sistemas_de_representa_o_de_conhecimento_e_racioc
_nio}
{\emph{Sistemas de Representação de Conhecimento e Raciocínio}}, ao longo do segundo semestre do 
terceiro ano do
\href{http://miei.di.uminho.pt}{Mestrado Integrado em Engenharia Informática} da 
\href{https://www.uminho.pt}{Universidade
do Minho}.

O presente trabalho tem como principal objetivo aprimorar a utilização da extensão à
programação em lógica, bem como a representação de conhecimento imperfeito. Assim, pretende-se 
desenvolver um sistema de representação de conhecimento e raciocínio
com capacidade para caracterizar um universo de discurso na área da vacinação global da 
população portuguesa no contexto
da pandemia COVID-19 que estamos a viver.

Por fim, é apresentada uma reflexão crítica sobre o trabalho desenvolvido, bem como os 
resultados obtidos.
\end{abstract}

\pagebreak

\tableofcontents

\pagebreak

\lstlistoflistings

\listoftables

\pagebreak

\pagenumbering{arabic}

\pagestyle{fancy}
\fancyhf{}

\rfoot{Página \thepage \hspace{1pt} de \pageref{LastPage}}

\renewcommand{\headrulewidth}{0pt}

\section{Introdução}

O presente relatório documenta a segunda fase do trabalho prático da Unidade Curricular de 
\href{https://miei.di.uminho.pt/plano_estudos.html#sistemas_de_representa_o_de_conhecimento_e_racioc
_nio}
{\emph{Sistemas de Representação de Conhecimento e Raciocínio}}. Este projeto tem como 
finalidade o desenvolvimento de um sistema de representação de conhecimento
e raciocínio com capacidade para caracterizar um universo de discurso na área da 
vacinação global da população portuguesa no contexto
da pandemia COVID-19 que estamos a viver.

Inicialmente, começou-se por representar conhecimento positivo e negativo, devido à
possibilidade de se ter respostas verdadeiras, falsas e desconhecidas, sendo que o
conhecimento positivo já tinha sido abordado na primeira fase do trabalho prático.
De seguida, implementou-se a noção de conhecimento imperfeito, devido à presença
da resposta como o desconhecido, existindo três tipos diferentes deste tipo de conhecimento
incerto, impreciso e interdito. 

Para a representação de conhecimento imperfeito foi necessário criar predicados correspondentes 
aos três tipos de conhecimento imperfeito -- incerto, impreciso e interdito. 

Neste documento, serão abordados os novos predicados criados, assim como as
alterações feitas nos que transitaram da primeira fase do trabalho prático, terminando com uma 
análise crítica dos resultados do trabalho efetuado.

\pagebreak

\section{Preliminares}

Na primeira fase do trabalho prático realizado anteriormente, foi desenvolvido um sistema de 
representação de conhecimento e raciocínio capaz de representar conhecimento perfeito, 
assemelhando-se a um sistema de bases de dados, na
medida em que assumia-se que a informação representada seria única e válida e que as
entidades representadas seriam as únicas existentes no mundo exterior.

Após um estudo mais aprofundado em relação à programação em lógica e à respetiva 
representação de informação, fomos confrontados com algumas limitações. Para debater essas 
limitações, importa ter em
atenção que este paradigma se baseia nos seguintes pressupostos:

\begin{itemize}
    \item \textbf{Pressuposto do Mundo Fechado}, que afirma que toda a informação que não é 
mencionada na base de conhecimento é considerada como falsa;
    \item \textbf{Pressuposto dos Nomes Únicos}, que garante que duas constantes diferentes (que 
definem valores atómicos ou objetos) designam duas entidades diferentes do universo de discurso;
    \item \textbf{Pressuposto do Domínio Fechado}, que indica que não existem mais objetos no 
universo de discurso para além daqueles designados por constantes na base de conhecimento.
\end{itemize}

Estes pressupostos foram salvaguardados na primeira fase do trabalho prático, mas aplicando-os
no contexto do mundo real podem trazer vários problemas. Por exemplo, num problema
simples do dia a dia como a intenção de atravessar uma estrada, é necessário ter em atenção
se está algum carro a aproximar-se. Para tal, recorrendo à programação em lógica, caso na
base de conhecimento não exista informação relativa a carros a aproximarem-se, o
motor de inferência indicará que não se aproxima nenhum automóvel e que,
consequentemente, é seguro atravessar. Porém, isto não prova que não se aproxima
nenhum carro, apenas que não há nenhuma prova de que se aproximam carros.

Desta forma, surge a necessidade de abolir alguns desses pressupostos visto que nem sempre
se pretende assumir que a informação representada é a única que é válida e, muito menos,
que as entidades representadas sejam as únicas existentes no mundo exterior.
Assim sendo, o Pressuposto dos Nomes Únicos continuará a ser válido uma vez que não interfere 
negativamente na representação do conhecimento. No entanto, o Pressuposto do Mundo Fechado e o 
Pressuposto do Domínio Fechado serão substituídos por:

\begin{itemize}
    \item \textbf{Pressuposto do Mundo Aberto}, que indica que podem existir outros factos ou 
conclusões verdadeiras para além daqueles representados na base de conhecimento;
    \item \textbf{Pressuposto do Domínio Aberto}, que garante que podem existir mais objetos do 
universo de discurso para além daqueles designados pelas constantes da base de conhecimento.
\end{itemize}

Para além dos novos pressupostos mencionados, a Programação em Lógica
Estendida permite também representar informação incompleta. Assim,
existem agora três diferentes tipos de conclusões para uma dada questão: \textit{verdadeira},
quando existe uma prova explícita de que se trata de conhecimento verdadeiro, \textit{falsa},
quando existe uma prova explícita de que se trata de conhecimento falso e \textit{desconhecido},
quando não existe informação que permita inferir uma ou outra das conclusões anteriores.

Por outro lado, o objetivo de estender a
Programação em Lógica é também passar a ser permitido representar informação negativa
explicitamente -- \textit{negação forte}. Desta forma, a extensão de um programa em lógica passa
agora a contar com dois tipos de negação -- a \textit{negação por falha na prova}, que é 
representada pelo
termo \texttt{nao}, que nos indica que não existe uma prova na base de conhecimento que
responda à questão e a \textit{negação forte}, que é uma forma de identificar informação 
negativa
ou falsa, representada pela conectiva \texttt{-}, que indica que existe uma prova na base de
conhecimento de que a questão é falsa.

\subsection{Representação de Conhecimento Imperfeito}

O conhecimento imperfeito consiste na ausência de informação em relação a uma questão 
colocada. Em caso de informação incompleta, isto é, desconhecida, existem três casos distintos:

\begin{itemize}
    \item \textbf{Incerto} -- representa valores nulos que não pertencem a nenhum
    conjunto determinado de valores, sendo portanto completamente desconhecidos;
    \item \textbf{Impreciso} -- representa valores nulos que pertencem a um conjunto
    determinado de valores (conhecimento desconhecido mas dentro de um conjunto finito de 
hipóteses), \textit{i.e} existe uma noção do que é falso mas não
    se conhece qual a verdadeira resposta;
    \item \textbf{Interdito} -- representa valores nulos desconhecidos e que não serão permitidos 
conhecer.
\end{itemize}

Em suma, considerando o trabalho desenvovlido na fase anterior, para a elaboração do presente 
trabalho prático resta agora desenvolver um
sistema de inferência que permita implementar os
mecanismos de raciocínio mencionados, inerentes a estes sistemas, assim como povoar a base de 
conhecimento com os diferentes tipo de conhecimento supramencionados.

\pagebreak

\section{Descrição do Trabalho e Análise de Resultados}

Nesta secção irá ser discutida e devidamente fundamentada toda a resolução desta segunda fase 
do trabalho prático. O foco estará em respeitar as necessidades das várias funcionalidades 
enumeradas no enunciado do trabalho prático, no sentido de se conseguir produzir uma resolução 
atenta, mas
concisa e aceitável.

\subsection{Base de Conhecimento}

Tendo em conta que o universo que se pretende representar continua a focar-se na vacinação global 
da população portuguesa no contexto da pandemia COVID-19, a base de conhecimento que foi 
desenvolvida
na primeira fase do trabalho prático acaba por se manter
praticamente igual, existindo apenas mínimas alterações.

Com o intuito de representar informação incompleta, torna-se necessário alterar o domínio de 
soluções passando a incluir o valor de verdade \textit{Desconhecido}. Este
representa um valor que não pode ser definido de imediato e é com ele que surge o
conceito de programação em lógica estendida.

Com esta alteração, a definição da base de conhecimento passa a ser a seguinte:

\begin{itemize}
    \item \texttt{utente: \#Idutente, No Seguranca\_Social, Nome, Data\_Nasc, Email, Telefone,} 
\texttt{Morada, Profissao,
    [Doencas\_Cronicas], \#CentroSaude -> \{V, F, D\}} -- Um uten-te tem associado a si um ID,
    número de Segurança Social, nome, data de nascimento, \textit{email}, telefone, morada, 
profissão, eventuais doenças
    crónicas, assim como o ID do centro de saúde que frequenta;
    \item \texttt{staff: \#Idstaff, \#Idcentro, Nome, Email -> \{V, F, D\}} -- Um elemento do 
\textit{staff} do centro de saúde
    é caracterizado pelo seu ID, pelo ID do centro de saúde em que exerce funções, e pelo seu 
nome e \textit{email};
    \item \texttt{centro\_saude: \#Idcentro, Nome, Morada, Telefone, Email -> \{V, F, D\}} -- \\Um 
centro de saúde é identificado
    pelo seu ID, nome, morada, telefone e \textit{email};
    \item \texttt{vacinacao\_Covid: \#Idstaff, \#Idutente, Data, Vacina, Toma -> \{V, F, D\}} -- 
\\O registo de um ato de
    vacinação é caracterizado pelo ID do elemento do \textit{staff} responsável pela 
administração da vacina, o ID do utente,
    a data em que e qual a vacina que a vacina foi administrada, assim como a respetiva toma.
    \item \texttt{medico: \#Idmedico, \#Idcentro, Nome, Email, Especialidade -> \{V, F, D\}} -- 
\\Um médico 
    possui também um ID, nome, \textit{email} e especialidade. Adicionalmente, existe uma 
associação entre o
    médico e o centro de saúde em que este exerce funções;
    \item \texttt{consulta: \#Idmedico, \#Idutente, \#Idcentro, Data -> \{V, F, D\}} -- Uma 
consulta é caracterizada pelo ID
    do médico, pelo ID do utente, o centro de saúde é realizada e a respetiva data;
    \item \texttt{tratamento: \#IdStaff, \#Idutente, \#Idcentro, Data, Descricao -> \{V, F, D\}} -- 
Um tratamento é caracterizado
    por uma uma breve descrição, o centro de saúde onde é realizado, assim como o elemento do 
\textit{staff} que responsável,
    o utente, e a data em que este foi prestado.
\end{itemize}

Como forma de construir uma base de conhecimento ampla e generalizada,
optamos por incluir informação negativa explicitamente, bem como explicitar diretamente o
Pressuposto do Mundo Fechado para alguns predicados. Desta forma, mantivemos este pressuposto
para os predicados \texttt{utente}, \texttt{staff} e \texttt{medico}, assumindo então que um 
utente, um elemento do \textit{staff} do centro de saúde 
um médico não definidos na base de conhecimento nem que contenham alguma exceção
associada são considerados conhecimento falso. Esta decisão traz bastante vantagens, como
por exemplo, no caso da representação de conhecimento imperfeito impreciso, visto que se
não se soube exatamente o valor da idade do utente mas se souber o intervalo de
valores em que esta se situa, é possível afirmar com certezas que o utente com a idade não 
contida entre esses valores é falso, apesar dessa informação não se encontrar
explicitamente na base de conhecimento.


\

\begin{lstlisting}[caption={Pressuposto do Mundo Fechado para os predicados \texttt{utente}, 
\texttt{centro\_saude} e \texttt{staff}}]
% Pressuposto do mundo fechado para o predicado utente
-utente(ID, NUM, NOME, DN, EMAIL, TLF, M, P, DC, CS) :-
    nao(utente(ID, NUM, NOME, DN, EMAIL, TLF, M, P, DC, CS)),
    nao(excecao(utente(ID, NUM, NOME, DN, EMAIL, TLF, M, P, DC, CS))).

% Pressuposto do mundo fechado para o predicado centro_saude
-centro_saude(ID, NOME, M, TLF, EMAIL) :-
    nao(centro_saude(ID, NOME, M, TLF, EMAIL)),
    nao(excecao(centro_saude(ID, NOME, M, TLF, EMAIL))).

% Pressuposto do mundo fechado para o predicado staff
-staff(IDS, IDCENTRO, NOME, EMAIL) :-
    nao(staff(IDS, IDCENTRO, NOME, EMAIL)),
    nao(excecao(staff(IDS, IDCENTRO, NOME, EMAIL))).
\end{lstlisting}


\pagebreak

\subsection{Representação de Conhecimento Perfeito}

A representação de conhecimento positivo já se encontra realizada na primeira fase do trabalho 
prático. No
entanto, para a realização desta segunda fase foi necessário acrescentar a
representação do conhecimento negativo.

\subsubsection{Conhecimento Perfeito Positivo}

Tendo o panorama de conhecimento bem definido, começou-se por povoar o sistema acrescentando 
factos perfeitos positivos relativos a todos os predicados mencionados anteriormente. O povoamento 
da base de conhecimento com conhecimento perfeito positivo encontra-se em anexo, na secção 
\ref{sec:perfeito_positivo}

\subsubsection{Conhecimento Perfeito Negativo}

O conhecimento perfeito negativo foi representado tendo em
conta os diferentes tipos.

\paragraph{Negação por Falha na Prova} -- Quando não existe nenhuma prova aquando da negação 
do predicado. Esta negação é representada pelo meta-predicado \texttt{nao}, que se
encontra definido em anexo (secção \ref{sec:aux}).

\paragraph{Negação Forte} -- Quando se afirma que um determinado predicado é falso. Esta 
negação é representada pelo teorema \texttt{-}. 

\

\begin{lstlisting}[caption={Povoamento da base de conhecimento com conhecimento perfeito negativo}]
-utente(13, 41582, 'Duarte Pedro', date(1987, 6, 2), 'duartepedro@gmail.com', 913654782, 'Coimbra', 
'Jornalista', [], 3).
-utente(14, 45217, 'Tiago Loureiro', date(1993, 5, 12), 'tiagoloureiro@gmail.com', 912396755, 
'Lisboa', 'Enfermeiro', [], 4).

-centro_saude(5, 'Centro Hospitalar Povoa de Varzim', 'Povoa de Varzim', 252690600, 
'centrohospitalarpovoadevarzim@gmail.com').

-staff(11, 3, 'Igor Rodrigues', 'igorrodrigues@gmail.com').
-staff(12, 2, 'Ana Mendes', 'anamendes@gmail.com').

-vacinacao_Covid(3, 5, date(2021, 3, 21), 'Pfizer', 1).
-vacinacao_Covid(3, 5, date(2021, 4, 2), 'Pfizer', 2).

-medico(7, 4, 'Cesar Martins', 'cesarmartins@gmail.com', 'Oftalmologia').

-consulta(3, 1, 4, date(2021,2,15)).

-tratamento(7, 5, 1, date(2021,2,17), 'Radiografia').
\end{lstlisting}

\pagebreak

\subsection{Representação de Conhecimento Imperfeito}

Na problemática da representação de informação incompleta, os valores nulos
surgem com o objetivo de distinguir situações em que as respostas às questões são
conhecidas e as situações em que as respostas às questões são
desconhecidas. É neste sentido que surgem os três tipos de conhecimento imperfeito -- incerto, 
impreciso e interdito -- 
que são essenciais para todo um alargar de possibilidades de respostas às questões com base nas 
fontes de conhecimento existentes.


\subsubsection{Conhecimento Imperfeito Incerto}

Este tipo de conhecimento corresponde ao conhecimento para o qual não existe nenhuma prova que 
comprove que este seja verdadeiro ou falso. Trata-se, portanto, de um valor nulo do tipo 
desconhecido e não necessariamente de um conjunto de
valores. 

Desta forma, o conhecimento imperfeito incerto diz respeito a conhecimento desconhecido de
um determinado campo do predicado, dentro de um conjunto ilimitado de hipóteses.

\subsubsection*{Utente}

No nosso caso de estudo, isto pode ser caracterizado pela situação em que se sabe que
existe um utente com ID igual a 15, chamado Mateus, mas que não se sabe qual a sua idade, uma vez 
que a sua data de nascimento é desconhecida. Neste caso, isto é definido como conhecimento 
incerto visto que não se sabe qual é a sua data de nascimento, nem se tem uma ideia de qual possa 
ser. Para
representar este tipo conhecimento, é necessário criar o utente mencionado com
\texttt{data\_desconhecida} no campo relativo à data de nascimento e criar uma exceção que 
indique que quando se questionar a base de conhecimento relativamente ao utente com 
\texttt{data\_desconhecida}, o resultado deverá ser \textit{desconhecido}.

\

\begin{lstlisting}[caption={Conhecimento Imperfeito Incerto relativo ao utente}]
% Nao se conhece a idade do utente #15 uma vez que a sua data de nascimento e desconhecida
utente(15, 56257, 'Mateus Silva', data_desconhecida, 'mateussilva@gmail.com', 915698401,
       'Coimbra', 'Bombeiro', [], 4).
excecao(utente(ID, NUM, NOME, DN, EMAIL, TLF, M, P, DC, CS)) :-
        utente(ID, NUM, NOME, data_desconhecida, EMAIL, TLF, M, P, DC, CS).

\end{lstlisting}

\subsubsection*{Centro de Saúde}

Relativamente ao centro de saúde, o conhecimento imperfeito incerto surge quando, por exemplo, 
não se conhece o número de telefone. 
De forma análoga, é necessário criar o centro de saúde mencionado com
\texttt{telefone\_desconhecido} no campo relativo ao telefone e criar uma exceção que indique que 
quando se questionar a base de conhecimento relativamente ao centro de saúde com 
\texttt{telefone\_desconhecido}, o resultado deverá ser \textit{desconhecido}.

\

\begin{lstlisting}[caption={Conhecimento Imperfeito Incerto relativo ao centro de saúde}]
% Nao se conhece o telefone do centro de saude #6
centro_saude(6, 'Hospital da Luz', telefone_desconhecido, 'Guimaraes',
             'hospitaldaluz@gmail.com').
excecao(centro_saude(ID, NOME, M, TLF, EMAIL)) :- 
    centro_saude(ID, NOME, M, telefone_desconhecido, EMAIL).
\end{lstlisting}

\subsubsection*{\textit{Staff}}

No que diz respeito ao \textit{staff} de um centro de saúde, o conhecimento imperfeito incerto 
surge quando, por exemplo, não se 
conhece o ID do centro de saúde em que um que funcionário exerce funções. Desta forma, é 
necessário criar o funcionário mencionado com
\texttt{idc\_desconhecido} no campo relativo ao ID do centro em que este exerce funções e criar 
uma exceção que indique que quando
se questionar a base de conhecimento relativamente ao funcionário com \texttt{idc\_desconhecido}, 
o resultado deverá ser \textit{desconhecido}.

\

\begin{lstlisting}[caption={Conhecimento Imperfeito Incerto relativo ao \textit{staff} do centro de 
saúde}]
% Nao se sabe em que centro de saude o Antonio exerce funcoes
staff(13, idc_desconhecido, 'Antonio Gomes', 'antoniogomes@gmail.com').
excecao(staff(IDS, IDCENTRO, NOME, EMAIL)) :- 
    staff(IDS, idc_desconhecido, NOME, EMAIL).
\end{lstlisting}

\subsubsection*{Vacinação}

O conhecimento imperfeito incerto surge associado aos registos de vacinação quando, por exemplo, 
não se conhece com que vacina
um determinado utente foi vacinado. Desta forma, é necessário criar o registo de vacinação 
mencionado com
\texttt{vacina\_desconhecida} no campo relativo à vacina e criar uma exceção que indique que 
quando
se questionar a base de conhecimento relativamente a um registo de vacinação com 
\texttt{vacina\_desconhecida}, o resultado deverá ser \textit{desconhecido}.

\

\begin{lstlisting} [caption={Conhecimento Imperfeito Incerto relativo aos registos de vacinação}]
% Nao se sabe com que vacina o utente #7 foi vacinado
vacinacao_Covid(1, 7, date(2020,3,21), vacina_desconhecida, 1).
excecao(vacinacao_Covid(STAFF, UTENTE, DATA, VACINA, TOMA)) :-
    vacinacao_Covid(STAFF, UTENTE, DATA, vacina_desconhecida, TOMA).
\end{lstlisting}


\subsubsection*{Médico}

O conhecimento imperfeito incerto surge associado ao médico quando, por exemplo, não se conhece a 
sua especialidade. Assim, à semelhança do que foi referido anteriormente, é necessário criar o 
médico mencionado com
\texttt{especialidade\_desconhecida} no campo relativo à especialidade médica e criar uma 
exceção que indique que quando se questionar a base de conhecimento relativamente a um médico 
com \texttt{especialidade\_desconhecida}, o resultado deverá ser \textit{desconhecido}.


\

\begin{lstlisting} [caption={Conhecimento Imperfeito Incerto relativo ao médico}]
% Nao se conhece a especialidade do medico #8
medico(8, 2, 'Henrique Ferreira', 'henriqueferreira@gmail',
       especialidade_desconhecida).
excecao(medico(ID, IDCENTRO, NOME, EMAIL, ESP)) :-
    medico(ID, IDCENTRO, NOME, EMAIL, especialidade_desconhecida).
\end{lstlisting}

\subsubsection*{Consulta}

O conhecimento imperfeito incerto surge associado a uma consulta quando, por exemplo, não se 
conhece qual o ID do utente que foi à consulta. Desta forma, é necessário criar o registo da 
consulta com
\texttt{utente\_desconhecido} no campo relativo ao utente e criar uma exceção que indique que 
quando se questionar a base de conhecimento relativamente a uma consulta com 
\texttt{utente\_desconhecido}, o resultado deverá ser \textit{desconhecido}.

\

\begin{lstlisting} [caption={Conhecimento Imperfeito Incerto relativo à consulta}]
% Nao se sabe qual o utente que foi a consulta
consulta(4, utente_desconhecido, 2, date(2021,2,15)).
excecao(consulta(IDM, IDU, IDC, DATA)) :-
    consulta(IDM, utente_desconhecido, IDC, DATA).
\end{lstlisting}


\subsubsection*{Tratamento}

Seguindo um raciocínio idêntico, o conhecimento imperfeito incerto surge associado a um 
tratamento quando, por exemplo, não se conhece qual o ID do funcionário responsável pelo mesmo. 
Desta forma, é necessário criar o registo do tratamento com
\texttt{ids\_desconhecido} no campo relativo ao funcionário e criar uma exceção que indique que 
quando se questionar a base de conhecimento relativamente a um tratamento com 
\texttt{ids\_desconhecido}, o resultado deverá ser \textit{desconhecido}. 

\

\begin{lstlisting} [caption={Conhecimento Imperfeito Incerto relativo ao tratamento}]
% Nao se sabe qual o elemento do staff do centro de saude responsavel pelo tratamentos
tratamento(ids_desconhecido, 1, 4, date(2020,11,4), 'Biopsia').
excecao(tratamento(IDS, IDU, IDC, DATA, DSCR)) :-
    tratamento(ids_desconhecido, IDU, IDC, DATA, DSCR).
\end{lstlisting}

\pagebreak

\subsubsection{Conhecimento Imperfeito Impreciso}

Tal como o conhecimento desconhecido, este tipo corresponde ao
conhecimento para o qual não há provas de que este seja verdadeiro ou falso.
No entanto, este insere-se num conjunto de valores conhecidos.

Assim, qualquer que seja a resposta ao predicado que utilize um valor fora
desse conjunto, o resultado será \textit{desconhecido}, uma vez que apenas se sabe que o
valor se encontra contido dentro desse conjunto.

\subsubsection*{Utente}

Neste caso de estudo temos, por exemplo, a situação de sabermos que existe
um utente com ID igual a 9, chamado Pedro Reis, mas que não sabemos se mora em Braga ou no Porto. 
De forma a contornar esta situação, através do uso de
exceções, podemos construir essas duas alternativas de modo a que quando a base de
conhecimento for questionada relativamente a este utente a resposta seja \textit{desconhecido} se a
morada for Braga ou Porto, e a resposta ser \textit{falso}, caso a
base de conhecimento seja questionada com a especialidade diferente destas.

\

\begin{lstlisting} [caption={Conhecimento Imperfeito Impreciso relativo ao utente}]
 % Nao se sabe se a Teresa mora em Braga ou no Porto
excecao(utente(16, 59240, 'Teresa Soares', date(1998, 1, 23), 'teresasoares@gmail.com', 913654700, 
'Braga', 'Estudante',
               ['Hipertensao'], 1)).
excecao(utente(16, 59240, 'Teresa Soares', date(1998, 1, 23), 'teresasoares@gmail.com', 913654700, 
'Porto', 'Estudante',
               ['Hipertensao'], 1)).
\end{lstlisting}

\subsubsection*{Centro de Saúde}

O conhecimento imperfeito impreciso surge associado ao centro de saúde quando, por exemplo, se 
sabe que existe um centro de saúde mas não se conhece a sua morada. A título de exemplo, 
considere-se o Hospital De Santiago, que não se sabe se se situa em Setúbal ou Lisboa. De forma a 
contornar esta situação, através do uso de exceções, podemos construir essas duas alternativas 
de modo a que quando a base de
conhecimento for questionada relativamente a este centro saúde a resposta seja 
\textit{desconhecido} se a
morada for Setúbal ou Lisboa, e a resposta ser \textit{falso}, caso a
base de conhecimento seja questionada com uma morada diferente destas.

\pagebreak

\begin{lstlisting} [caption={Conhecimento Imperfeito Impreciso relativo ao centro de saúde}]
% Nao se sabe se o centro de saude #7 esta localizado em Setubal ou em Lisboa
excecao(centro_saude(7, 'Hospital De Santiago', 'Setubal', 265509200,
        'hospitaldesantiago@gmail.com')).
excecao(centro_saude(7, 'Hospital De Santiago', 'Lisboa', 265509200,
        'hospitaldesantiago@gmail.com')).
\end{lstlisting}

\subsubsection*{\textit{Staff}}

Neste caso de estudo temos, por exemplo, a situação de sabermos que existe
um funcionário com ID igual a 14, chamado Jorge Carvalho, mas que não sabemos se exerce funções 
no centro de saúde \#2 ou \#3. De forma a contornar esta situação, através do uso de
exceções, podemos construir essas duas alternativas de modo a que quando a base de
conhecimento for questionada relativamente a este funcionário a resposta seja 
\textit{desconhecido} se o ID do centro de saúde for 2 ou 3, e a resposta ser \textit{falso}, caso 
a
base de conhecimento seja questionada com um ID diferente.

\

\begin{lstlisting} [caption={Conhecimento Imperfeito Impreciso relativo ao \textit{staff} do centro 
de saúde}]
% Nao se sabe se o Jorge exerce funcoes no centro de saude #2 ou no #3
excecao(staff(14, 2, 'Jorge Carvalho', 'jorgecarvalho@gmail.com')).
excecao(staff(14, 3, 'Jorge Carvalho', 'jorgecarvalho@gmail.com')).
\end{lstlisting}


\subsubsection*{Vacinação}

Neste caso de estudo temos, por exemplo, a situação de sabermos que um utente foi vacinado, mas 
que não sabemos com que vacina -- apenas se sabe que ou foi com a vacina da \textit{Pfizer} ou com 
a da \textit{AstraZeneca}. De forma a contornar esta situação, através do uso de
exceções, podemos construir essas duas alternativas de modo a que quando a base de
conhecimento for questionada relativamente a este registo de vacinação seja \textit{desconhecido} 
se a vacina utilizada for da \textit{Pfizer} ou da \textit{AstraZeneca}, e a resposta ser 
\textit{falso}, caso a
base de conhecimento seja questionada com uma vacina diferente.

\

\begin{lstlisting} [caption={Conhecimento Imperfeito Impreciso relativo aos registos de 
vacinação}]
excecao(vacinacao_Covid(3, 11, date(2021, 3, 21), 'Pfizer', 1)).
excecao(vacinacao_Covid(3, 11, date(2021, 3, 21), 'Astrazeneca', 1)).
\end{lstlisting}

\subsubsection*{Médico}

No presente caso de estudo, temos, por exemplo, a situação de sabermos que existe
um médico com ID igual a 9, chamado Pedro Reis, mas que não sabemos se é especialista em
Dermatologia ou em Cardiologia. De forma a contornar esta situação, através do uso de
exceções, podemos construir essas duas alternativas de modo a que quando a base de
conhecimento for questionada relativamente a este médico a resposta seja \textit{desconhecido} se a
especialidade for Dermatologia ou Cardiologia, e a resposta ser \textit{falso}, caso a
base de conhecimento seja questionada com a especialidade diferente destas.

\

\begin{lstlisting} [caption={Conhecimento Imperfeito Impreciso relativo ao médico}]
% Nao se sabe se o medico #9 e dermatologista ou cardiologista
excecao(medico(9, 1, 'Pedro Reis', 'pedroreis@gmail.com',
        'Dermatologia')).
excecao(medico(9, 1, 'Pedro Reis', 'pedroreis@gzmail.com',
       'Cardiologia')).
\end{lstlisting}

\subsubsection*{Consulta}

No caso de estudo apresentado, temos, também, a situação de sabermos que o médico \#8 deu uma 
consulta, mas não sabemos se o foi o utente \#5 ou \#8 que foi à consulta. De forma a contornar 
esta situação, através do uso de
exceções, podemos construir essas duas alternativas de modo a que quando a base de
conhecimento for questionada relativamente a esta consulta, a resposta seja \textit{desconhecido} 
se a
o ID do utente for 5 ou 8, e a resposta ser \textit{falso} caso a
base de conhecimento seja questionada com um ID diferente.

\

\begin{lstlisting} [caption={Conhecimento Imperfeito Impreciso relativo à consulta}]
% Nao se sabe se foi o utente #5 ou #8 que foi a consultas
excecao(consulta(4, 5, 2, date(2021,3,1))).
excecao(consulta(4, 8, 2, date(2021,3,1))).
\end{lstlisting}

\subsubsection*{Tratamento}

No caso de estudo apresentado, temos, por fim, a situação de sabermos que o o utente \#8 realizou 
uma prova de esforço, mas não sabemos se o funcionário responsável foi o funcinário \#3 ou 
\#5. De forma a contornar esta situação, através do uso de
exceções, podemos construir essas duas alternativas de modo a que quando a base de
conhecimento for questionada relativamente a este tratamento, a resposta seja \textit{desconhecido} 
se a
o ID do funcionário for 3 ou 5, e a resposta ser \textit{falso} caso a
base de conhecimento seja questionada com um ID diferente.

\

\begin{lstlisting} [caption={Conhecimento Imperfeito Impreciso relativo ao tratamento}]
% Nao se sabe se foi o elemento do staff do centro de saude #3 ou #5 responsavel pelo tratamento
excecao(tratamento(3, 8, 2, date(2020,12,15), 'Prova de Esforco')).
excecao(tratamento(5, 8, 2, date(2020,12,15), 'Prova de Esforco')).
\end{lstlisting}

\

Por fim, importa salientar ainda que apesar de termos criado exemplos para todos os
predicados, apenas é sentida uma diferença significativa na resposta por parte do sistema
de inferência no caso de predicados assentes no PMF -- \texttt{utente}, \texttt{staff} e 
\texttt{medico}. Apenas nestes
casos é que o sistema de inferência dá \textit{falso} como resultado caso a base de
conhecimento não seja questionada com as alternativas existentes nas exceções
construídas, e dá \textit{desconhecido} caso seja questionada com uma das alternativas. 

\subsubsection{Conhecimento Imperfeito Interdito}

Este último tipo de conhecimento, para além de caracterizar um tipo de dados
desconhecido, caracteriza também, um tipo de dados que não se admite que surja na
base de conhecimento.
Nesta situação, o valor nulo para além de identificar um valor desconhecido,
representará um valor que não é permitido especificar ou conhecer e qualquer
tentativa para a concretizar será rejeitada como sendo provocadora de inconsistência
na informação presente na base de conhecimento.

Desta forma, este tipo de conhecimento, para além de
identificar os valores desconhecidos, não permite que haja evolução desse conhecimento, isto é, 
não
é permitido especificar ou conhecer o seu valor.

\subsubsection*{Utente}

Para mostrar este tipo de conhecimento criou-se um utente chamado Diogo, identificado pelo ID 17, 
do qual
não se conhece o número de Segurança Social -- representado por \texttt{sc\_desconhecido}. Como 
neste tipo de conhecimento
é impossível alguma vez descobrir o número de Segurança Social utente, torna-se necessário 
criar um invariante que não permita
a evolução deste conhecimento (abordado na secção \ref{sec:inv_interdito}) e para isso, 
criou-se um predicado nulo para identificar o \texttt{sc\_desconhecido}.
Desta forma, ao procurar por um utente, se existir o predicado \texttt{nulo(sc\_desconhecido)}, 
nunca vai ser
possível evoluir as informações relativas ao número de Segurança Social deste utente.

\

\begin{lstlisting} [caption={Conhecimento Imperfeito Interdito relativo ao utente}]
% Nao se sabe nem e possivel saber qual o numero de Seguranca Social do utente #17
utente(17, sc_desconhecido, 'Diogo Fernandes', date(1995, 9, 5),
       'diogofernandes@gmail.com', 924581465, 'Lisboa', [], 4).
nulo(sc_desconhecido).
excecao(utente(ID, _, NOME, DN, EMAIL, TLF, M, P, DC, CS)) :-
    utente(ID, sc_desconhecido, NOME, DN, EMAIL, TLF, M, P, DC, CS).
\end{lstlisting}

\subsubsection*{Centro de Saúde}

De forma análoga, neste caso de estudo temos, por exemplo, a situação de existir um centro de 
saúde do qual não se conhece o telefone -- representado por \texttt{tlf\_desconhecido}. Como 
neste tipo de conhecimento
é impossível alguma vez descobrir o telefone do centro de saúde, torna-se necessário criar um 
invariante que não permita a evolução deste conhecimento (abordado secção 
\ref{sec:inv_interdito}) e para isso, criou-se um predicado nulo para identificar o 
\texttt{tlf\_desconhecido}.
Desta forma, ao procurar por um centro de saúde, se existir o predicado 
\texttt{nulo(tlf\_desconhecido)}, nunca vai ser
possível evoluir as informações relativas ao telefone deste centro de saúde.

\pagebreak

\begin{lstlisting} [caption={Conhecimento Imperfeito Interdito relativo ao centro de saúde}]
% Nao se sabe nem e possivel saber o telefone do centro de saude #8
centro_saude(8, 'Hospital da Misericordia de Evora', 'Evora',
             tlf_desconhecido,
             'hospitaldamisericordiadeevora@gmail.com').
nulo(tlf_desconhecido).
excecao(centro_saude(ID, NOME, M, _, EMAIL)) :-
    centro_saude(ID, NOME, M, tlf_desconhecido, EMAIL).
\end{lstlisting}

\subsubsection*{\textit{Staff}}

Neste caso de estudo temos, por exemplo, a situação de existir um funcionário de um centro de 
saúde do qual não se conhece o email -- representado por \texttt{email\_desconhecido}. Como neste 
tipo de conhecimento
é impossível alguma vez descobrir o \textit{email} do funcionário, torna-se necessário criar um 
invariante que não permita a evolução deste conhecimento (abordado secção 
\ref{sec:inv_interdito}) e para isso, criou-se um predicado nulo para identificar o 
\texttt{email\_desconhecido}.
Assim, ao procurar por um funcionário, se existir o predicado \texttt{nulo(email\_desconhecido)}, 
nunca vai ser
possível evoluir as informações relativas ao \textit{email} deste funcionário.

\

\begin{lstlisting} [caption={Conhecimento Imperfeito Interdito relativo ao \textit{staff} do centro 
de saúde}]
% Nao se conhece nem e possivel conhecer o email do Rafael
staff(15, 1, 'Rafael Costa', email_desconhecido).
nulo(email_desconhecido).
excecao(staff(IDS, IDCENTRO, NOME, _)) :-
    staff(IDS, IDCENTRO, NOME, email_desconhecido).
\end{lstlisting}

\subsubsection*{Vacinação}

Pode também a situação de existir registo de vacinação do qual não se conhece qual a vacina 
administrada -- representada por \texttt{vac\_desconhecida}. Como neste tipo de conhecimento
é impossível alguma vez descobrir qual a vacina administrada, torna-se necessário criar um 
invariante que não permita a evolução deste conhecimento (abordado secção 
\ref{sec:inv_interdito}) e para isso, criou-se um predicado nulo para identificar a 
\texttt{vac\_desconhecida}.
Assim, ao procurar por um registo de vacinação, se existir o predicado 
\texttt{nulo(vac\_desconhecida)}, nunca vai ser
possível evoluir as informações relativas ao registo de vacinação.

\

\begin{lstlisting} [caption={Conhecimento Imperfeito Interdito relativo aos registos de 
vacinação}]
% Nao se conhece nem e possivel conhecer qual a vacina administrada
vacinacao_covid(3, 12, date(2021, 7, 17), vac_desconhecida, 1).
nulo(vac_desconhecida).
excecao(vacinacao_Covid(STAFF, UTENTE, DATA, _, TOMA)) :-
    vacinacao_Covid(STAFF, UTENTE, DATA, vac_desconhecida, TOMA).
\end{lstlisting}

\subsubsection*{Médico}

Considere-se também a situação de existir um médico do qual não se conhece qual o centro de 
saúde em que exerce funções -- representado por \texttt{centro\_desconhecido}. Como neste tipo 
de conhecimento
é impossível alguma vez descobrir qual o centro de saúde em que este exerce funções, torna-se 
necessário criar um invariante que não permita a evolução deste conhecimento (abordado secção 
\ref{sec:inv_interdito}) e para isso, criou-se um predicado nulo para identificar o 
\texttt{centro\_desconhecido}.
Assim, ao procurar por um médico, se existir o predicado \texttt{nulo(centro\_desconhecido)}, 
nunca vai ser
possível evoluir as informações relativas ao médico.

\

\begin{lstlisting} [caption={Conhecimento Imperfeito Interdito relativo ao médico}]
% Nao se sabe nem e possivel saber em que centro de saude o medico #10 exerce funcoes
medico(10, centro_desconhecido, 'Rui Costa', 'ruicosta@gmail.com',
       'Otorrinolaringologia').
nulo(centro_desconhecido).
excecao(medico(ID, _, NOME, EMAIL, ESP)) :-
    medico(ID, centro_desconhecido, NOME, EMAIL, ESP).
\end{lstlisting}

\subsubsection*{Consulta}

Neste caso de estudo temos, por exemplo, a situação de não se saber qual o utente que foi a uma 
consulta -- representado por \texttt{ut\_desconhecido}. Como neste tipo de conhecimento
é impossível alguma vez descobrir qual o utente em questão, torna-se necessário criar um 
invariante que não permita a evolução deste conhecimento (abordado secção 
\ref{sec:inv_interdito}) e para isso, criou-se um predicado nulo para identificar o 
\texttt{ut\_desconhecido}.
Assim, ao procurar por uma consulta, se existir o predicado \texttt{nulo(ut\_desconhecido)}, nunca 
vai ser
possível evoluir as informações relativas à consulta em questão.

\

\begin{lstlisting} [caption={Conhecimento Imperfeito Interdito relativo à consulta}]
% Nao se sabe nem e possivel saber o utente que foi a consulta
consulta(4, ut_desconhecido, 2, date(2021,1,15)).
nulo(ut_desconhecido).
excecao(consulta(IDM, _, IDC, DATA)) :-
    consulta(IDM, ut_desconhecido, IDC, DATA).
\end{lstlisting}

\subsubsection*{Tratamento}

Por fim, existe também a situação de não se saber qual o centro de saúde em que um tratamento 
foi realizado -- representado por \texttt{cs\_desconhecido}. Como neste tipo de conhecimento
é impossível alguma vez descobrir qual o centro de saúde em questão, torna-se necessário criar 
um invariante que não permita a evolução deste conhecimento (abordado secção 
\ref{sec:inv_interdito}) e para isso, criou-se um predicado nulo para identificar o 
\texttt{cs\_desconhecido}.
Assim, ao procurar por um tratamento, se existir o predicado \texttt{nulo(cs\_desconhecido)}, nunca 
vai ser
possível evoluir as informações relativas à consulta em questão.

\

\begin{lstlisting} [caption={Conhecimento Imperfeito Interdito relativo ao tratamento}]
% Nao se conhece nem e possivel conhecer qual o centro de saude em que o tratamento foi realizado
tratamento(7, 5, cs_desconhecido, date(2020,11,20), 'Exame Pulmonar').
nulo(cs_desconhecido).
excecao(tratamento(IDS, IDU, IDC, _, DSCR)) :-
    tratamento(IDS, IDU, cs_desconhecido, DATA, DSCR).
\end{lstlisting}

\pagebreak

\subsection{Invariantes}

Para que a Base de Conhecimento funcione corretamente, é indispensável garantir certas 
condições responsáveis pelo controlo da inserção e remoção de conhecimento.

Para esse efeito, foi desenvolvido, já na primeira fase do trabalho prático, uma série de 
invariantes responsáveis por este controlo que optaremos por não tornar a incluir neste 
relatório.

Contudo, para além deste controlo da inserção e remoção, acaba por ser também imperativo 
garantir igualmente o controlo da:

\begin{itemize}
    \item Inserção de conhecimento perfeito positivo e negativo;
    \item Inserção de conhecimento imperfeito interdito.
\end{itemize}

\subsubsection{Invariantes Universais}

Existem alguns invariantes que podem ser generalizados e, desta forma, todos os
predicados que serão inseridos na nossa base de conhecimento terão que obedecer a esses
invariantes.
Em primeiro lugar, construímos invariantes que garantem que não existe
conhecimento perfeito positivo e conhecimento perfeito negativo repetido na nossa base de
conhecimento, ou seja, que não existe conhecimento redundante, tal como se pode ver de seguida.

\

\begin{lstlisting}[caption={Invariantes relativos ao conhecimento repetido}]
% Invariante que garante que nao existe conhecimento perfeito positivo repetido
+T :: (solucoes(T, T, R),
       comprimento(R, 1)).

% Invariante que garante que nao existe conhecimento perfeito negativo repetido
+(-T) :: (solucoes(T, -T, R),
          comprimento(R, 1)).
\end{lstlisting}

\

Por outro lado, foram também criados invariantes que não permitem que seja adicionado 
conhecimento perfeito positivo que contradiz conhecimento perfeito negativo
presente na base de conhecimento e vice-versa.
Estes invariantes, apresentados de seguida, são essenciais para que a informação existente seja 
coerente e não seja contraditória.

\

\begin{lstlisting}[caption={Invariante relativos ao conhecimento contraditório}]
% Invariante que nao permite adicionar conhecimento perfeito positivo que contradiz conhecimento 
perfeito negativo
+T :: nao(-T).

% Invariante que nao permite adicionar conhecimento perfeito negativo que contradiz conhecimento 
perfeito positivo
+(-T) :: nao(T).
\end{lstlisting}

\subsection{Invariantes Estruturais e Referenciais}

Os invariantes são completamente indispensáveis para um correto funcionamento
de todo o sistema. Apenas com a sua introdução é possível efetuar um controlo da informação 
que inserida e removida, criando,
assim, um intermediário necessário para que os predicados de evolução e retrocesso funcionem 
como ambicionado.

Para esta gestão de informação se poder processar da maneira mais adequada e lógica face ao 
sistema, foram criados, na primeira fase do trabalho prático, invariantes
associados tanto à inserção como à remoção de conhecimento.

Importa salientar que, em relação ao trabalho desenvolvido na primeira fase, foi acrescentado um 
invariante que permite a validação da data de nascimento do utente, sendo para que o definir foi 
necessário definir primeiro o predicado \texttt{verificaData}, apresentado em anexo (secção 
\ref{sec:aux}). Além disso, foram desenvolvidos invariantes relativos ao conhecimento perfeito 
negativo, assim como predicados no sentido de não permitir a evolução de conhecimento imperfeito 
interdito.

\input{invariantes}

\subsection{Invariantes Associados ao Conhecimento Imperfeito Interdito}
\label{sec:inv_interdito}

\subsubsection*{Utente}

\

\begin{lstlisting}
% Invariante que impede a insercao de conhecimento perfeito positivo relativo a um utente com 
numero de Seguranca Social interdito (conhecimento imperfeito interdito)
+utente(ID, NUM, NOME, DN, EMAIL, TLF, M, P, DC, CS) :: (
    solucoes((ID, NUM, NOME, DN, EMAIL, TLF, M, P, DC, CS),
             (utente(ID, NUM, NOME, DN, EMAIL, TLF, M, P, DC, CS),
              nulo(NUM)), S),
    comprimento(S, N),
    N == 0
).
\end{lstlisting}

\pagebreak

\subsection{Problemática da Evolução do Conhecimento}

O tratamento da problemática da evolução do conhecimento prende-se com o facto de manter a base 
de conhecimento coesa e inviolável em termos
de existência de conhecimento repetido, tendo em conta cada inserção ou remoção que possa 
acontecer.

Para que fosse possível a inserção e remoção de conhecimento na base de conhecimento, foram 
desenvolvidos os predicados evolução e involução. Estes predicados obrigam a que a inserção 
ou remoção cumpra certas regras definidas pelos invariantes.

Assim, é necessário implementar uma série de medidas que não permita:

\begin{itemize}
    \item Remover informação que seja dependente de outra;
    \item Inserção de informação repetida.
\end{itemize}

De modo a garantir esta segurança, não pode ser apagada informação
dependente de outra, ou seja, não pode ser removida a informação sobre
um utente se este possuir registos de vacinação a si associados. Além disso, não pode
ser adicionada informação repetida, uma vez que não traz adição de conhecimento.
Assim, no momento de alteração de informação, é necessário testar se esta
corrompe a base de conhecimento. Estes testes são efetuados através do uso
de invariantes, previamente explicados

\subsection{Evolução de Conhecimento}

\subsubsection{Conhecimento Perfeito Positivo}

Numa primeira fase, e de maneira a todo o conhecimento positivo ser evoluído, ou
seja, ser adicionado à base de conhecimento, foi criado o seguinte predicado \texttt{evolucao}. 
Para ser possível adicionar conhecimento, este predicado verifica primeiro
se o elemento a ser adicionado respeita todos os invariantes e, caso seja verdade
então este é adicionado à base de conhecimento.

\

\begin{lstlisting}[caption={Evolução de Conhecimento Perfeito Positivo}]
% Extensao do meta-predicado evolucao: T -> {V, F}
% Conhecimento perfeito positivo
evolucao(T) :- solucoes(I, +T::I, L),
               insercao(T),
               teste(L).
\end{lstlisting}

\subsubsection{Conhecimento Perfeito Negativo}

De maneira a ser exequível a adição de conhecimento perfeito negativo na
base de conhecimento, o predicado criado é em tudo semelhante ao interior, mas neste caso
é necessário especificar que o conhecimento a adicionar é negativo. Neste caso,
os invariantes a respeitar são os que dizem respeito à inserção de conhecimento
negativo.

\pagebreak

\begin{lstlisting}[caption={Evolução de Conhecimento Perfeito Negativo}]
% Extensao do meta-predicado evolucao: T -> {V, F}
% Conhecimento perfeito negativo
evolucao(-T) :- solucoes(I, +(-T)::I, L),
                insercao(-T),
                teste(L).
\end{lstlisting}

\subsection{Involução de Conhecimento}

\subsubsection{Conhecimento Perfeito Positivo}

Foi criado o predicado \texttt{involucao} análogo ao das aulas práticas de forma a permitir a 
remoção de conhecimento da base de conhecimento, verificando todos os invariantes necessários. 
Assim, só em caso de sucesso é que o conhecimento é, de facto, removido.

\

\begin{lstlisting}[caption={Involução de Conhecimento Perfeito Positivo}]
% Extensao do meta-predicado involucao: T -> {V, F}
% Conhecimento perfeito positivo
involucao(T) :- solucoes(I, -T::I, L),
                teste(L),
                remocao(T).
\end{lstlisting}

\subsubsection{Conhecimento Perfeito Negativo}

De forma análoga, para permitir também a involução do conhecimento negativo, foi
criado outro predicado \texttt{involucao} mas com a particularidade de ser necessário especificar 
que se trata de conhecimento negativo.

\ 

\begin{lstlisting}[caption={Involução de Conhecimento Perfeito Negativo}]
% Extensao do meta-predicado involucao: T -> {V, F}
% Conhecimento perfeito negativo
involucao(-T) :- solucoes(I, -(-T)::I, L),
                 teste(L),
                 remocao(-T).
\end{lstlisting}

\pagebreak

\subsection{Sistema de Inferência}

Um sistema de inferência capaz de implementar os mecanismos de raciocínio
inerentes aos sistemas que anteriormente foram representados é um passo indispensável ser 
efetuado uma vez que é com este sistema que podemos "criar" uma
espécie de interpretador de questões.

Assim, a implementação do sistema de inferência começou pela criação do meta-predicado 
\texttt{demo: Questao, Resposta -> \{V, F\}}, que, para uma determinada questão, e coloca na
Resposta, um dos valores do conjunto de soluções --  verdadeiro, falso e desconhecido. 

A determinação da resposta sobre a questão colocada através do meta-predicado \texttt{demo} é 
verdadeira 
caso exista conhecimento positivo sobre a mesma, falsa caso exista conhecimento negativo, e
desconhecida caso não exista nenhum dos conhecimentos.

Em termos práticos, considerando uma questão abstrata $q(X)$, em que $X$ pode tomar a forma $X_1, 
X_2, ..., X_n$, definem-se as seguintes três respostas
possíveis para essa questão:

\begin{itemize}
    \item Verdadeiro se $\exists X \ q(X)$
    \item Falso se $\exists X \ \neg q(X)$
    \item Desconhecido caso contrário
\end{itemize}

Neste sentido, foi desenvolvido interpretador
de questões. Este interpretador de questões trata-se de um predicado que se
baseia nos diferentes tipos de conhecimento imperfeito.

O predicado \texttt{demo} apresenta uma resposta, mediante uma questão, determinando o seu valor 
de verdade.

\

\begin{lstlisting}[caption={Extensão do meta-predicado \texttt{demo}}, label={lst:demo}]
% Extensao do meta-predicado demo: Questao, Resposta -> {V, F, D}
demo(Questao, verdadeiro) :- Questao.
demo(Questao, falso) :- -Questao.
demo(Questao, desconhecido) :- nao(Questao), nao(-Questao).
\end{lstlisting}

Para complementar esse sistema, foram
construídos também predicados que permitem fazer conjunções e/ou disjunções de duas ou mais 
\textit{queries}.

\subsubsection*{Disjunção}

\begin{table}[H]
    \centering
    \begin{tabular}{|c|c|c|}
    \hline
    $P$          & $Q$          & $P \lor Q$                \\ \hline
    Verdadeiro   & -            & Verdadeiro                \\ \hline
    -            & Verdadeiro   & Verdadeiro                \\ \hline
    Falso        & Falso        & Falso                     \\ \hline
    Falso        & Desconhecido & Desconhecido              \\ \hline
    Desconhecido & Falso        & Desconhecido              \\ \hline
    Desconhecido & Desconhecido & Desconhecido              \\ \hline
    \end{tabular}
    \caption{Tabela de verdade da disjunção}
\end{table}

\subsubsection*{Conjunção}

\begin{table}[H]
    \centering
    \begin{tabular}{|c|c|c|}
    \hline
    $P$          & $Q$          & $P \land Q$               \\ \hline
    Verdadeiro   & Verdadeiro   & Verdadeiro                \\ \hline
    Verdadeiro   & Desconhecido & Desconhecido              \\ \hline
    Desconhecido & Verdadeiro   & Desconhecido              \\ \hline
    Desconhecido & Desconhecido & Desconhecido              \\ \hline
    Falso        & -            & Falso                     \\ \hline
    -            & Falso        & Falso                     \\ \hline
    \end{tabular}
    \caption{Tabela de verdade da conjunção}
\end{table}

Assim, com o auxílio dos predicados \texttt{disjuncao} e \texttt{conjuncao}, desenvolveu-se o 
predicado \texttt{demoComp}, que extende a funcionalidade do meta-predicado \texttt{demo}, 
permitindo interpretas uma lista de questões.

\

\begin{lstlisting}[caption={Extensão do predicado \texttt{disjuncao}}]
% Extensao do predicado disjuncao: X, Y, R -> {V, F, D}
disjuncao(verdadeiro, _, verdadeiro).
disjuncao(_, verdadeiro, verdadeiro).
disjuncao(falso, falso, falso).
disjuncao(falso, desconhecido, desconhecido).
disjuncao(desconhecido, falso, desconhecido).
disjuncao(desconhecido, desconhecido, desconhecido).
\end{lstlisting}

\

\begin{lstlisting}[caption={Extensão do predicado \texttt{conjuncao}}]
% Extensao do predicado conjuncao: X, Y, R -> {V, F, D}
conjuncao(verdadeiro, verdadeiro, verdadeiro).
conjuncao(verdadeiro, desconhecido, desconhecido).
conjuncao(desconhecido, verdadeiro, desconhecido).
conjuncao(desconhecido, desconhecido, desconhecido).
conjuncao(falso, _, falso).
conjuncao(_, falso, falso).
\end{lstlisting}

\

\begin{lstlisting}[caption={Extensão do predicado \texttt{demoComp}}]
% Extensao do predicado demoComp: Lista, R -> {V, F, D}
demoComp([Q], R) :- demo(Q,R).
demoComp([Q1, ou, Q2|T], R) :- demo(Q1, R1),
                               demoComp([Q2|T], R2),
                               disjuncao(R1, R2, R).
demoComp([Q1, e, Q2|T], R) :- demo(Q1, R1),
                              demoComp([Q2|T], R2),
                              conjuncao(R1,R2,R).
\end{lstlisting}

\pagebreak

\section{Conclusão e Trabalho Futuro}

No final deste trabalho prático, foram implementadas todas as funcionalidades requisitadas,
permitindo a representação de todo o tipo de conhecimento estudado, tanto perfeito (positivo, 
negativo) como imperfeito (incerto, impreciso e interdito), com vários
exemplos e demonstrações de cada um deles.

Contrariamente ao que foi feito na primeira fase do trabalho prático, nesta o principal foco foi, 
não apostar tanto na diversidade de predicados capazes de explorar o sistema em questão, mas 
garantir que se representava o conhecimento de forma completa,
exemplificando explicitamente, para os quatro predicados base do enunciado do problema,
todos os tipos de conhecimento estudados.

Como trabalho futuro, poderia abordar-se a problemática da evolução e involução de 
conhecimento imperfeito. Além disso, a base de conhecimento poderia ser mais estendida, ainda que 
o trabalho proposto seja
suficiente para demonstrar todos os tipos de conhecimento imperfeito bem como os invariantes
desenvolvidos para a gestão da base de conhecimento.

\pagebreak

\section{Anexos}

\subsection{Predicados Auxiliares}
\label{sec:aux}

Para o desenvolvimento dos procedimentos requeridos pelo sistema de representação de conhecimento 
e raciocínio, foi necessário recorrer a vários predicados auxiliares que
foram extremamente úteis no decorrer de todo o processo.
Em seguida, apresentam-se esses predicados desenvolvidos.

\subsubsection*{Meta-predicado \texttt{nao}}

O meta-predicado \texttt{nao} devolve o valor de verdade contrário ao termo \texttt{Q}
passado como parâmetro através da \textit{negação fraca}, isto é, caso exista uma prova 
afirmativa
ou negativa explícita de \texttt{Q} na base de conhecimento dá \texttt{no} como resposta e, caso 
contrário -- \textit{i.e.} na ausência de prova -- dá \texttt{yes}.

\

\begin{lstlisting}[caption={Extensão do meta-predicado \texttt{nao}}]
% Extensao do meta-predicado nao: Questao -> {V, F}
% Negacao fraca
nao(Q) :- Q, !, fail.
nao(_).
\end{lstlisting}

\subsubsection*{Predicado \texttt{comprimento}}

O predicado comprimento coloca em \texttt{R} o comprimento da lista passada como argumento.

\

\begin{lstlisting}[caption={Extensão do predicado \texttt{comprimento}}]
% Extensao do predicado comprimento: S, N -> {V, F}
comprimento(S, N) :- length(S, N).
\end{lstlisting}

\subsubsection*{Meta-predicado \texttt{solucoes}}

Utiliza-se este predicado quando se pretende obter a listagem de todas as soluções possíveis 
\texttt{Z}, para uma dada questão
\texttt{Y}, cujo formato da lista é especificado por \texttt{X}. Faz-se uso do predicado 
disponibilizado pelo \texttt{PROLOG},
\texttt{findall}, uma vez que este não falha na eventualidade de não existir resposta a esta 
questão, ao contrário do
que aconteceria com o predicado \texttt{bagof}.

\

\begin{lstlisting}[language=Prolog, caption={Extensão do meta-predicado \texttt{solucoes}}]
% Extensao do meta-predicado solucoes: X, Y, Z -> {V, F}
solucoes(X, Y, Z) :- findall(X, Y, Z).
\end{lstlisting}

\subsection*{Meta-predicado \texttt{teste}}

O meta-predicado \texttt{teste} testa se todos os predicados passados como parâmetro
são verdadeiros.

\

\begin{lstlisting}[caption={Extensão do meta-predicado \texttt{teste}}]
% Extensao do meta-predicado teste: L -> {V, F}
teste([]).
teste([H|T]) :- H, teste(T).
\end{lstlisting}

\subsection*{Meta-predicado \texttt{insercao} e \texttt{remocao}}

O meta-predicado \texttt{insercao} coloca \texttt{T} na base de conhecimento no caso de
sucesso, retornando \texttt{yes}, e retira \texttt{T} no caso de haver retrocesso, retornando 
\texttt{no}. O meta-
predicado \texttt{remocao}, por sua vez, faz o oposto, ou seja, remove \texttt{T} da
base de conhecimento no caso de sucesso, retornando \texttt{yes} e adiciona \texttt{T} no caso de 
haver retrocesso,
retornando \texttt{no}.

\

\begin{lstlisting}[caption={Extensão do meta-predicado \texttt{insercao}}]
% Extensao do meta-predicado insercao: T -> {V, F}
insercao(T) :- assert(T).
insercao(T) :- retract(T), !, fail.
\end{lstlisting}

\

\begin{lstlisting}[caption={Extensão do meta-predicado \texttt{remocao}}]
% Extensao do meta-predicado remocao: T -> {V, F}
remocao(T) :- retract(T).
remocao(T) :- assert(T), !, fail.
\end{lstlisting}


\subsection*{Predicado \texttt{verificaData}}

O predicado \texttt{verificaData} permite determinar se uma determinada data é válida.

\begin{lstlisting}[caption={Extensão do predicado \texttt{verificaData}}]
% Extensao do predicado verificaData: date(D, M, A) -> {V, F}
verificaData(date(D, 1, A)) :- D > 0, D =< 31, A > 0.
verificaData(date(D, 2, A)) :- D > 0, A mod 4 =:= 0, D =< 29, A > 0. 
verificaData(date(D, 2, A)) :- D > 0, A mod 4 =\= 0, D =< 28, A > 0.
verificaData(date(D, 3, A)) :- D > 0, D =< 31, A > 0.
verificaData(date(D, 4, A)) :- D > 0, D =< 30, A > 0.
verificaData(date(D, 5, A)) :- D > 0, D =< 31, A > 0.
verificaData(date(D, 6, A)) :- D > 0, D =< 30, A > 0.
verificaData(date(D, 7, A)) :- D > 0, D =< 31, A > 0.
verificaData(date(D, 8, A)) :- D > 0, D =< 31, A > 0.
verificaData(date(D, 9, A)) :- D > 0, D =< 30, A > 0.
verificaData(date(D, 10, A)) :- D > 0, D =< 31, A > 0.
verificaData(date(D, 11, A)) :- D > 0, D =< 30, A > 0.
verificaData(date(D, 12, A)) :- D > 0, D =< 31, A > 0.
\end{lstlisting}

\pagebreak

\subsection{Base de Conhecimento}

\subsubsection{Conhecimento Perfeito Positivo}
\label{sec:perfeito_positivo}

\

\input{conhecimento_perfeito_positivo}


\pagebreak

\nomenclature[01]{ID}{Identificador}
\nomenclature[02]{PMF}{Pressuposto do Mundo Fechado}

\printnomenclature

\pagebreak

\begin{thebibliography}{2}

\section*{Referências Bibliográficas}

\bibitem{analide}
ANALIDE, Cesar, Neves, José\\ 
"Representação de Informação Incompleta"\\
Texto Pedagógico, Grupo de Inteligência Artificial, Centro de Ciências e Tecnologias da 
Computação, Portugal, 1996

\bibitem{analide2}
ANALIDE, Cesar, NOVAIS, Paulo, Neves, José\\
"Sugestões para a Redacção de Relatórios Técnicos"\\
Relatório Técnico, Departamento de Informática, Universidade do Minho, Portugal, 2011

\end{thebibliography}

\end{document}
