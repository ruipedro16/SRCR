\section{Anexos}

\subsection{Predicados Auxiliares}
\label{sec:aux}

Para o desenvolvimento dos procedimentos requeridos pelo sistema de representação de conhecimento 
e raciocínio, foi necessário recorrer a vários predicados auxiliares que
foram extremamente úteis no decorrer de todo o processo.
Em seguida, apresentam-se esses predicados desenvolvidos.

\subsubsection*{Meta-predicado \texttt{nao}}

O meta-predicado \texttt{nao} devolve o valor de verdade contrário ao termo \texttt{Q}
passado como parâmetro através da \textit{negação fraca}, isto é, caso exista uma prova 
afirmativa
ou negativa explícita de \texttt{Q} na base de conhecimento dá \texttt{no} como resposta e, caso 
contrário -- \textit{i.e.} na ausência de prova -- dá \texttt{yes}.

\

\begin{lstlisting}[caption={Extensão do meta-predicado \texttt{nao}}]
% Extensao do meta-predicado nao: Questao -> {V, F}
% Negacao fraca
nao(Q) :- Q, !, fail.
nao(_).
\end{lstlisting}

\subsubsection*{Predicado \texttt{comprimento}}

O predicado comprimento coloca em \texttt{R} o comprimento da lista passada como argumento.

\

\begin{lstlisting}[caption={Extensão do predicado \texttt{comprimento}}]
% Extensao do predicado comprimento: S, N -> {V, F}
comprimento(S, N) :- length(S, N).
\end{lstlisting}

\subsubsection*{Meta-predicado \texttt{solucoes}}

Utiliza-se este predicado quando se pretende obter a listagem de todas as soluções possíveis 
\texttt{Z}, para uma dada questão
\texttt{Y}, cujo formato da lista é especificado por \texttt{X}. Faz-se uso do predicado 
disponibilizado pelo \texttt{PROLOG},
\texttt{findall}, uma vez que este não falha na eventualidade de não existir resposta a esta 
questão, ao contrário do
que aconteceria com o predicado \texttt{bagof}.

\

\begin{lstlisting}[language=Prolog, caption={Extensão do meta-predicado \texttt{solucoes}}]
% Extensao do meta-predicado solucoes: X, Y, Z -> {V, F}
solucoes(X, Y, Z) :- findall(X, Y, Z).
\end{lstlisting}

\subsection*{Meta-predicado \texttt{teste}}

O meta-predicado \texttt{teste} testa se todos os predicados passados como parâmetro
são verdadeiros.

\

\begin{lstlisting}[caption={Extensão do meta-predicado \texttt{teste}}]
% Extensao do meta-predicado teste: L -> {V, F}
teste([]).
teste([H|T]) :- H, teste(T).
\end{lstlisting}

\subsection*{Meta-predicado \texttt{insercao} e \texttt{remocao}}

O meta-predicado \texttt{insercao} coloca \texttt{T} na base de conhecimento no caso de
sucesso, retornando \texttt{yes}, e retira \texttt{T} no caso de haver retrocesso, retornando 
\texttt{no}. O meta-
predicado \texttt{remocao}, por sua vez, faz o oposto, ou seja, remove \texttt{T} da
base de conhecimento no caso de sucesso, retornando \texttt{yes} e adiciona \texttt{T} no caso de 
haver retrocesso,
retornando \texttt{no}.

\

\begin{lstlisting}[caption={Extensão do meta-predicado \texttt{insercao}}]
% Extensao do meta-predicado insercao: T -> {V, F}
insercao(T) :- assert(T).
insercao(T) :- retract(T), !, fail.
\end{lstlisting}

\

\begin{lstlisting}[caption={Extensão do meta-predicado \texttt{remocao}}]
% Extensao do meta-predicado remocao: T -> {V, F}
remocao(T) :- retract(T).
remocao(T) :- assert(T), !, fail.
\end{lstlisting}


\subsection*{Predicado \texttt{verificaData}}

O predicado \texttt{verificaData} permite determinar se uma determinada data é válida.

\begin{lstlisting}[caption={Extensão do predicado \texttt{verificaData}}]
% Extensao do predicado verificaData: date(D, M, A) -> {V, F}
verificaData(date(D, 1, A)) :- D > 0, D =< 31, A > 0.
verificaData(date(D, 2, A)) :- D > 0, A mod 4 =:= 0, D =< 29, A > 0. 
verificaData(date(D, 2, A)) :- D > 0, A mod 4 =\= 0, D =< 28, A > 0.
verificaData(date(D, 3, A)) :- D > 0, D =< 31, A > 0.
verificaData(date(D, 4, A)) :- D > 0, D =< 30, A > 0.
verificaData(date(D, 5, A)) :- D > 0, D =< 31, A > 0.
verificaData(date(D, 6, A)) :- D > 0, D =< 30, A > 0.
verificaData(date(D, 7, A)) :- D > 0, D =< 31, A > 0.
verificaData(date(D, 8, A)) :- D > 0, D =< 31, A > 0.
verificaData(date(D, 9, A)) :- D > 0, D =< 30, A > 0.
verificaData(date(D, 10, A)) :- D > 0, D =< 31, A > 0.
verificaData(date(D, 11, A)) :- D > 0, D =< 30, A > 0.
verificaData(date(D, 12, A)) :- D > 0, D =< 31, A > 0.
\end{lstlisting}

\pagebreak

\subsection{Base de Conhecimento}

\subsubsection{Conhecimento Perfeito Positivo}
\label{sec:perfeito_positivo}

\

\input{conhecimento_perfeito_positivo}
