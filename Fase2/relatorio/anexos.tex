\section{Anexos}

\subsection{Predicados Auxiliares}
\label{sec:aux}

Para o desenvolvimento dos procedimentos requeridos pelo sistema de representação de conhecimento 
e raciocínio, foi necessário recorrer a vários predicados auxiliares que
foram extremamente úteis no decorrer de todo o processo.
Em seguida, apresentam-se esses predicados desenvolvidos.

\subsubsection*{Meta-predicado \texttt{nao}}

O meta-predicado \texttt{nao} devolve o valor de verdade contrário ao termo \texttt{Q}
passado como parâmetro através da \textit{negação fraca}, isto é, caso exista uma prova 
afirmativa
ou negativa explícita de \texttt{Q} na base de conhecimento dá \texttt{no} como resposta e, caso 
contrário -- \textit{i.e.} na ausência de prova -- dá \texttt{yes}.

\

\begin{lstlisting}[caption={Extensão do meta-predicado \texttt{nao}}]
% Extensao do meta-predicado nao: Questao -> {V, F}
% Negacao fraca
nao(Q) :- Q, !, fail.
nao(_).
\end{lstlisting}

\subsubsection*{Predicado \texttt{comprimento}}

O predicado comprimento coloca em \texttt{R} o comprimento da lista passada como argumento.

\

\begin{lstlisting}[caption={Extensão do predicado \texttt{comprimento}}]
% Extensao do predicado comprimento: S, N -> {V, F}
comprimento(S, N) :- length(S, N).
\end{lstlisting}

\subsubsection*{Meta-predicado \texttt{solucoes}}

Utiliza-se este predicado quando se pretende obter a listagem de todas as soluções possíveis 
\texttt{Z}, para uma dada questão
\texttt{Y}, cujo formato da lista é especificado por \texttt{X}. Faz-se uso do predicado 
disponibilizado pelo \texttt{PROLOG},
\texttt{findall}, uma vez que este não falha na eventualidade de não existir resposta a esta 
questão, ao contrário do
que aconteceria com o predicado \texttt{bagof}.

\

\begin{lstlisting}[language=Prolog, caption={Extensão do meta-predicado \texttt{solucoes}}]
% Extensao do meta-predicado solucoes: X, Y, Z -> {V, F}
solucoes(X, Y, Z) :- findall(X, Y, Z).
\end{lstlisting}

\subsection*{Meta-predicado \texttt{teste}}

O meta-predicado \texttt{teste} testa se todos os predicados passados como parâmetro
são verdadeiros.

\

\begin{lstlisting}[caption={Extensão do meta-predicado \texttt{teste}}]
% Extensao do meta-predicado teste: L -> {V, F}
teste([]).
teste([H|T]) :- H, teste(T).
\end{lstlisting}

\subsection*{Meta-predicado \texttt{insercao} e \texttt{remocao}}

O meta-predicado \texttt{insercao} coloca \texttt{T} na base de conhecimento no caso de
sucesso, retornando \texttt{yes}, e retira \texttt{T} no caso de haver retrocesso, retornando 
\texttt{no}. O meta-
predicado \texttt{remocao}, por sua vez, faz o oposto, ou seja, remove \texttt{T} da
base de conhecimento no caso de sucesso, retornando \texttt{yes} e adiciona \texttt{T} no caso de 
haver retrocesso,
retornando \texttt{no}.

\

\begin{lstlisting}[caption={Extensão do meta-predicado \texttt{insercao}}]
% Extensao do meta-predicado insercao: T -> {V, F}
insercao(T) :- assert(T).
insercao(T) :- retract(T), !, fail.
\end{lstlisting}

\

\begin{lstlisting}[caption={Extensão do meta-predicado \texttt{remocao}}]
% Extensao do meta-predicado remocao: T -> {V, F}
remocao(T) :- retract(T).
remocao(T) :- assert(T), !, fail.
\end{lstlisting}


\subsection*{Predicado \texttt{verificaData}}

O predicado \texttt{verificaData} permite determinar se uma determinada data é válida.

\begin{lstlisting}[caption={Extensão do predicado \texttt{verificaData}}]
% Extensao do predicado verificaData: date(D, M, A) -> {V, F}
verificaData(date(D, 1, A)) :- D > 0, D =< 31, A > 0.
verificaData(date(D, 2, A)) :- D > 0, A mod 4 =:= 0, D =< 29, A > 0. 
verificaData(date(D, 2, A)) :- D > 0, A mod 4 =\= 0, D =< 28, A > 0.
verificaData(date(D, 3, A)) :- D > 0, D =< 31, A > 0.
verificaData(date(D, 4, A)) :- D > 0, D =< 30, A > 0.
verificaData(date(D, 5, A)) :- D > 0, D =< 31, A > 0.
verificaData(date(D, 6, A)) :- D > 0, D =< 30, A > 0.
verificaData(date(D, 7, A)) :- D > 0, D =< 31, A > 0.
verificaData(date(D, 8, A)) :- D > 0, D =< 31, A > 0.
verificaData(date(D, 9, A)) :- D > 0, D =< 30, A > 0.
verificaData(date(D, 10, A)) :- D > 0, D =< 31, A > 0.
verificaData(date(D, 11, A)) :- D > 0, D =< 30, A > 0.
verificaData(date(D, 12, A)) :- D > 0, D =< 31, A > 0.
\end{lstlisting}

\pagebreak

\subsection{Base de Conhecimento}

\subsubsection{Conhecimento Perfeito Positivo}
\label{sec:perfeito_positivo}

\

\begin{lstlisting}[caption={Povoamento inicial da Base de Conhecimento com Conhecimento Perfeito Positivo}]
% Extensao do predicado utente: #Idutente, No Seguranca_Social, Nome,
                                Data_Nasc, Email, Telefone, Morada,
                                Profissao, [Doencas_Cronicas],
                                #CentroSaude -> {V, F, D}
                                
utente(1, 14492, 'Duarte Carvalho', date(1998,4,21),
       'duartecarvalho@gmail.com', 91761416, 'Lisboa', 'Estudante',
      ['Asma'], 4).
utente(2, 95110, 'Mariana Pereira', date(1988,11,12),
       'marianapereira@gmail.com', 915992520, 'Porto', 'Atleta', [], 4).      
utente(3, 79297, 'Teresa Marques', date(1987,8,30),
       'teresamarques@gmail.com', 913844675, 'Viseu', 'Engenheira',
       ['Hipertensao'], 4).      
utente(4, 71723, 'Sofia Soares', date(1979,6,8), 'sofiasoares@gmail.com',
       919555582, 'Porto', 'Advogada', ['Diabetes'], 2).      
utente(5, 40203, 'Rui Rocha', date(1987,12,24), 'ruirocha@gmail.com',
       919219565, 'Braga', 'Atleta', [], 1).      
utente(6, 77645, 'Joao Martins', date(1998,11,12), 'joaomartins@gmail.com
       ', 917630282, 'Coimbra', 'Estudante', ['Asma'], 2).      
utente(7, 67275, 'Diogo Rodrigues', date(1995,1,25),
       'diogorodrigues@gmail.com', 916543686, 'Lisboa', 'Engenheiro',
       ['Diabetes','Cancro'], 2).      
utente(8, 76991, 'Francisca Lopes', date(1986,9,15),
       'franciscalopes@gmail.com', 913672965, 'Braga', 'Enfermeira',
       ['Asma'], 2).
utente(9, 82539, 'Tiago Lima', date(1978,12,12), 'tiagolima@gmail.com',
       91683448, 'Lisboa', 'Medico', ['Hipertensao'], 5).      
utente(10, 16086, 'Daniela Macedo', date(1977,10,25),
       'danielamacedo@gmail.com', 911216397, 'Coimbra', 'Medico', [], 3).
utente(11, 56418, 'Daniel Santos', date(1955, 2, 23),
       'danielsantos@gmail.com', 915632478, 'Faro', 'Advogado', [], 4).
    
`\pagebreak`
       
% Extensao do predicado centro_saude: #Idcentro, Nome, Morada, Telefone,
                                      Email -> {V, F, D}
                                      
centro_saude(1, 'Hospital de Braga', 'Braga', 253027000,
             'hospitaldebraga@gmail.com').
centro_saude(2, 'Hospital Da Senhora Da Oliveira', 'Guimaraes',
             253540330, 'hospitaldasenhoradaoliveira@gmail.com').
centro_saude(3, 'Centro Hospitalar e Universitario de Coimbra',
             'Coimbra', 239400400,
             'centrohospitalareuniversitariodecoimbra@gmail.com').
centro_saude(4, 'Hospital de Santa Maria', 'Lisboa', 217805000,
             'hospitaldesantamaria@gmail.com').
             
             
% Extensao do predicado staff: #Idstaff, #Idcentro, Nome, 
                               Email -> {V, F, D}
                               
staff(1, 4, 'Teresa Rodrigues', 'teresarodrigues@gmail.com').
staff(2, 3, 'Diogo Martins', 'diogomartins@gmail.com').
staff(3, 2, 'Daniela Marques', 'danielamarques@gmail.com').
staff(4, 2, 'Joao Lopes', 'joaolopes@gmail.com').
staff(5, 3, 'Rui Lima', 'ruilima@gmail.com').
staff(6, 4, 'Mariana Soares', 'marianasoares@gmail.com').
staff(7, 1, 'Duarte Pereira', 'duartepereira@gmail.com').
staff(8, 4, 'Francisca Lima', 'franciscalima@gmail.com').
staff(9, 3, 'Sofia Macedo', 'sofiamacedo@gmail.com').
staff(10, 1, 'Tiago Carvalho', 'tiagocarvalho@gmail.com').


% Extensao do predicado vacinacao_Covid: #Idstaff, #Idutente, Data,
                                         Vacina, Toma -> {V, F, D}
                                         
vacinacao_Covid(7, 1, date(2021,4,23), 'Pfizer', 1).
vacinacao_Covid(5, 1, date(2021,8,24), 'Pfizer', 2).
vacinacao_Covid(7, 2, date(2020,2,1), 'Astrazeneca', 1).
vacinacao_Covid(10, 3, date(2020,10,13), 'Pfizer', 1).
vacinacao_Covid(9, 4, date(2020,11,11), 'Pfizer', 1).
vacinacao_Covid(6, 4, date(2020,12,19), 'Pfizer', 2).
vacinacao_Covid(1, 5, date(2020,3,21), 'Astrazeneca', 1).
vacinacao_Covid(5, 5, date(2020,7,30), 'Astrazeneca', 2).
vacinacao_Covid(8, 6, date(2020,5,3), 'Pfizer', 1).
vacinacao_Covid(7, 6, date(2020,10,20), 'Pfizer', 2).
vacinacao_Covid(7, 8, date(2021,7,17), 'Pfizer', 1).
vacinacao_Covid(9, 9, date(2020,2,21), 'Astrazeneca', 1).
vacinacao_Covid(3, 10, date(2020,9,29), 'Pfizer', 1).
vacinacao_Covid(1, 10, date(2020,12,13), 'Pfizer', 2).

`\pagebreak`

% Extensao do predicado medico: #Idmedico, #Idcentro, Nome, Email,
                                Especialidade -> {V, F, D}
                                
medico(1, 3, 'Goncalo Santos', 'goncalosantos@gmail.com', 'Cardiologia').
medico(2, 3, 'Roberto Moreira', 'robertomoreira@gmail.com',
       'Anestesiologia').
medico(3, 4, 'Rui Santos', 'ruisantos@gmail.com', 'Clinica Geral').
medico(4, 2, 'Ines Nunes', 'inesnunes@gmail.com', 'Dermatologia').
medico(5, 1, 'Hugo Alves', 'hugoalves@gmail.com', 'Gastrenterologia').
medico(6, 1, 'Ricardo Sousa', 'ricardosousa@gmail.com',
       'Medicina Dentaria').
       
% Extensao do predicado consulta: #Idmedico, #Idutente, #Idcentro, 
                                  Data -> {V, F, D}
                                  
consulta(3, 1, 4, date(2020,10,15)).
consulta(6, 5, 1, date(2021,3,2)).
consulta(4, 8, 2, date(2020,12,20)).
consulta(3, 2, 4, date(2020,8,14)).

% Extensao do predicado tratamento: #IdStaff, #Idutente, #Idcentro, Data,
                                    Descricao -> {V, F, D}
                                    
tratamento(4, 4, 2, date(2021,3,14), 'Radiografia Perna').
tratamento(6, 1, 4, date(2021,2,1), 'Eletrocardiograma').
tratamento(7, 5, 1, date(2020,5,14), 'Exame Pulmonar').
tratamento(3, 8, 2, date(2021,3,1), 'Analises Clinicas').
\end{lstlisting}

