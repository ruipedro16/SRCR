\subsection{Problemática da Evolução do Conhecimento}

O tratamento da problemática da evolução do conhecimento prende-se com o facto de manter a base 
de conhecimento coesa e inviolável em termos
de existência de conhecimento repetido, tendo em conta cada inserção ou remoção que possa 
acontecer.

Para que fosse possível a inserção e remoção de conhecimento na base de conhecimento, foram 
desenvolvidos os predicados evolução e involução. Estes predicados obrigam a que a inserção 
ou remoção cumpra certas regras definidas pelos invariantes.

Assim, é necessário implementar uma série de medidas que não permita:

\begin{itemize}
    \item Remover informação que seja dependente de outra;
    \item Inserção de informação repetida.
\end{itemize}

De modo a garantir esta segurança, não pode ser apagada informação
dependente de outra, ou seja, não pode ser removida a informação sobre
um utente se este possuir registos de vacinação a si associados. Além disso, não pode
ser adicionada informação repetida, uma vez que não traz adição de conhecimento.
Assim, no momento de alteração de informação, é necessário testar se esta
corrompe a base de conhecimento. Estes testes são efetuados através do uso
de invariantes, previamente explicados

\subsection{Evolução de Conhecimento}

\subsubsection{Conhecimento Perfeito Positivo}

Numa primeira fase, e de maneira a todo o conhecimento positivo ser evoluído, ou
seja, ser adicionado à base de conhecimento, foi criado o seguinte predicado \texttt{evolucao}. 
Para ser possível adicionar conhecimento, este predicado verifica primeiro
se o elemento a ser adicionado respeita todos os invariantes e, caso seja verdade
então este é adicionado à base de conhecimento.

\

\begin{lstlisting}[caption={Evolução de Conhecimento Perfeito Positivo}]
% Extensao do meta-predicado evolucao: T -> {V, F}
% Conhecimento perfeito positivo
evolucao(T) :- solucoes(I, +T::I, L),
               insercao(T),
               teste(L).
\end{lstlisting}

\subsubsection{Conhecimento Perfeito Negativo}

De maneira a ser exequível a adição de conhecimento perfeito negativo na
base de conhecimento, o predicado criado é em tudo semelhante ao interior, mas neste caso
é necessário especificar que o conhecimento a adicionar é negativo. Neste caso,
os invariantes a respeitar são os que dizem respeito à inserção de conhecimento
negativo.

\pagebreak

\begin{lstlisting}[caption={Evolução de Conhecimento Perfeito Negativo}]
% Extensao do meta-predicado evolucao: T -> {V, F}
% Conhecimento perfeito negativo
evolucao(-T) :- solucoes(I, +(-T)::I, L),
                insercao(-T),
                teste(L).
\end{lstlisting}

\subsection{Involução de Conhecimento}

\subsubsection{Conhecimento Perfeito Positivo}

Foi criado o predicado \texttt{involucao} análogo ao das aulas práticas de forma a permitir a 
remoção de conhecimento da base de conhecimento, verificando todos os invariantes necessários. 
Assim, só em caso de sucesso é que o conhecimento é, de facto, removido.

\

\begin{lstlisting}[caption={Involução de Conhecimento Perfeito Positivo}]
% Extensao do meta-predicado involucao: T -> {V, F}
% Conhecimento perfeito positivo
involucao(T) :- solucoes(I, -T::I, L),
                teste(L),
                remocao(T).
\end{lstlisting}

\subsubsection{Conhecimento Perfeito Negativo}

De forma análoga, para permitir também a involução do conhecimento negativo, foi
criado outro predicado \texttt{involucao} mas com a particularidade de ser necessário especificar 
que se trata de conhecimento negativo.

\ 

\begin{lstlisting}[caption={Involução de Conhecimento Perfeito Negativo}]
% Extensao do meta-predicado involucao: T -> {V, F}
% Conhecimento perfeito negativo
involucao(-T) :- solucoes(I, -(-T)::I, L),
                 teste(L),
                 remocao(-T).
\end{lstlisting}