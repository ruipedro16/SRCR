\subsubsection{Invariantes Associados à Inserção de Conhecimento}

No que diz respeito à inserção de conhecimento, foram criados invariantes que impedem a inserção de conhecimento repetido.
Adicionalmente, foi necessário construír um conjunto de invariantes que não possibilitasse a inserção de informação impossível
de se introduzir na base de conhecimento em questão.

\subsubsection*{Utente}
Considerou-se que os IDs não se poderiam de forma alguma repetir uma vez que identificam univocamente cada um dos utentes.
Para além disso, foi tido em consideração o facto de estes serem representados por números inteiros. Por outro lado, é
necessário garantir que o centro de saúde que o utente frequenta existe de facto. Por fim, foi tido em consideração o facto
de não poder existir mais do que um utente com mesmo número de Segurança Social. Todas estas restrições são impostas
pelos seguintes invariantes:

\

\begin{lstlisting}[caption={Invariantes de inserção relativos ao predicado \texttt{utente}}]
% O ID do utente deve ser um numero inteiro e deve ser unico
+utente(ID, _, _, _, _, _, _, _, _, _) :: (
    integer(ID),
    solucoes(ID, utente(ID, _, _, _, _, _, _, _, _, _), S),
    comprimento(S, N),
    N == 1
).

% O ID do utente deve ser um numero inteiro e deve ser unico
% Conhecimento perfeito negativo
+(-utente(ID, _, _, _, _, _, _, _, _, _)) :: (
    integer(ID),
    solucoes(ID, utente(ID, _, _, _, _, _, _, _, _, _), S),
    comprimento(S, N),
    N == 1
).

% O utente tem de estar associado a um centro de saude que exista
+utente(_, _, _, _, _, _, _, _, _, CS) :: (
    solucoes(CS, centro_saude(CS, _, _, _, _), S),
    comprimento(S, N),
    N == 1
).

% O utente tem de estar associado a um centro de saude que exista
% Conhecimento perfeito negativo
+(-utente(_, _, _, _, _, _, _, _, _, CS)) :: (
    solucoes(CS, centro_saude(CS, _, _, _, _), S),
    comprimento(S, N),
    N == 1
).

% So pode existir um utente com um determinado numero de Seguranca Social
+utente(_, NUM, _, _, _, _, _, _, _, _) :: (
    solucoes(NUM, utente(_, NUM, _, _, _, _, _, _, _, _), S),
    comprimento(S, N),
    N == 1
).

% So pode existir um utente com um determinado numero de Seguranca Social
% Conhecimento perfeito negativo
+(-utente(_, NUM, _, _, _, _, _, _, _, _)) :: (
    solucoes(NUM, utente(_, NUM, _, _, _, _, _, _, _, _), S),
    comprimento(S, N),
    N == 1
).

% A data de nascimento do utente deve ser uma data valida
+utente(_, _, _, DN, _, _, _, _, _, _) :: (verificaData(DN)).

% A data de nascimento do utente deve ser uma data valida
% Conhecimento perfeito negativo
+(-utente(_, _, _, DN, _, _, _, _, _, _)) :: (verificaData(DN)).

% Invariante que impede a insercao de conhecimento perfeito positivo relativo a um utente com numero de Seguranca Social interdito (conhecimento imperfeito interdito)
+utente(ID, NUM, NOME, DN, EMAIL, TLF, M, P, DC, CS) :: (
    solucoes((ID, NUM, NOME, DN, EMAIL, TLF, M, P, DC, CS),
             (utente(ID, NUM, NOME, DN, EMAIL, TLF, M, P, DC, CS),
              nulo(NUM)), S),
    comprimento(S, N),
    N == 0
).

\end{lstlisting}

\subsubsection*{\textit{Staff}}

Seguindo um raciocínio idêntico, os IDs dos elementos do \textit{staff} de um centro de saúde, além de serem representados
por números inteiros, não se podem repetir. Além disso é necessário garantir que o centro de saúde em que o elemento do
\textit{staff} presta serviços existe de facto, restrições essas que são impostas pelos seguintes invariantes.

\

\begin{lstlisting}[caption={Invariantes de inserção relativos ao predicado \texttt{staff}}]
% O ID de cada elemento do staff do centro de saude deve ser um numero inteiro e deve ser unico
+staff(ID, _, _, _) :: (
    integer(ID),
    solucoes(ID, staff(ID, _, _, _), S),
    comprimento(S, N),
    N == 1
).

% O ID de cada elemento do staff do centro de saude deve ser um numero inteiro e deve ser unico
% Conhecimento perfeito negativo
+(-staff(ID, _, _, _)) :: (
    integer(ID),
    solucoes(ID, staff(ID, _, _, _), S),
    comprimento(S, N),
    N == 1
).

% Cada elemento do staff do centro de saude tem de estar associado a um centro de saude que exista
+staff(_, CS, _, _) :: (
    solucoes(CS, centro_saude(CS, _, _, _, _), S),
    comprimento(S, N),
    N == 1
).

% Cada elemento do staff do centro de saude tem de estar associado a um centro de saude que exista
% Conhecimento perfeito negativo
+(-staff(_, CS, _, _)) :: (
    solucoes(CS, centro_saude(CS, _, _, _, _), S),
    comprimento(S, N),
    N == 1
).
`\pagebreak`
% Invariante que impede a insercao de conhecimento perfeito positivo relativo a um funcionario com email interdito (conhecimento imperfeito interdito)
+staff(IDS, IDCENTRO, NOME, EMAIL) :: (
    solucoes((IDS, IDCENTRO, NOME, EMAIL),
              (staff(IDS, IDCENTRO, NOME, EMAIL), nulo(EMAIL)), S),
    comprimento(S, N),
    N == 0
).

\end{lstlisting}

\subsubsection*{Centro de Saúde}
Uma vez que o ID, representado por um número inteiro, identifica univocamente um centro de saúde, não é possível a existência
de mais do que um centro de saúde com o mesmo ID, restrição essa que é imposta pelo seguinte invariante:

\

\begin{lstlisting}[caption={Invariante de inserção relativo ao predicado \texttt{centro\_saude}}]
% O ID do centro de saude deve ser um numero inteiro e deve ser unico
+centro_saude(ID, _, _, _, _) :: (
    integer(ID),
    solucoes(ID, centro_saude(ID, _, _, _, _), S),
    comprimento(S, N),
    N == 1
).

% O ID do centro de saude deve ser um numero inteiro e deve ser unico
% Conhecimento perfeito negativo
+(-centro_saude(ID, _, _, _, _)) :: (
    integer(ID),
    solucoes(ID, centro_saude(ID, _, _, _, _), S),
    comprimento(S, N),
    N == 1
).

% Invariante que impede a insercao de conhecimento perfeito positivo relativo a um centro de saude com telefone interdito (conhecimento imperfeito interdito)
+centro_saude(ID, NOME, M, TLF, EMAIL) :: (
    solucoes((ID, NOME, M, TLF, EMAIL),
             (centro_saude(ID, NOME, M, TLF, EMAIL),
              nulo(TLF)), S),
    comprimento(S, N),
    N == 0
).
\end{lstlisting}

\subsubsection*{Vacinação}
De forma a que um registo de vacinação seja válido, este tem de estar associado a um elemento do \textit{staff} e a um
utente que, de facto, existam na base de conhecimento. Além disso, a toma da vacina só pode assumir os valores 1 e 2,
sendo estritamente necessário que a segunda toma ocorra após a primeira. Note-se também que não é permitida a inserção de
registos repetidos. Todas estas restrições são impostas pelos seguintes invariantes:

\

\begin{lstlisting}[caption={Invariantes de inserção relativos ao predicado \texttt{vacinacao\_covid}}]
% Nao permite a insercao de registos duplicados
+vacinacao_Covid(STAFF, UTENTE, DATA, VACINA, TOMA) :: (
    solucoes((STAFF, UTENTE, DATA, VACINA, TOMA),
              vacinacao_Covid(STAFF, UTENTE, DATA, VACINA, TOMA), S),
    comprimento(S, N),
    N == 1
).

% Nao permite a insercao de registos duplicados
% Conhecimento perfeito negativo
+(-vacinacao_Covid(STAFF, UTENTE, DATA, VACINA, TOMA)) :: (
    solucoes((STAFF, UTENTE, DATA, VACINA, TOMA),
              vacinacao_Covid(STAFF, UTENTE, DATA, VACINA, TOMA), S),
    comprimento(S, N),
    N == 1
).

% O registo de vacinacao tem de estar associado a um elemento do staff que exista
+vacinacao_covid(IDS, _, _, _) :: (
    solucoes(IDS, staff(IDS, _, _, _), S),
    comprimento(S, N),
    N == 1
).

% O registo de vacinacao tem de estar associado a um elemento do staff que exista
% Conhecimento perfeito negativo
+(-vacinacao_covid(IDS, _, _, _)) :: (
    solucoes(IDS, staff(IDS, _, _, _), S),
    comprimento(S, N),
    N == 1
).

% O registo de vacinacao tem de estar associado a um utente que existe
+vacinacao_covid(_, IDU, _, _, _) :: (
    solucoes(IDU, utente(IDU, _, _, _, _, _, _, _, _, _), S),
    comprimento(S, N),
    N == 1
).
`\pagebreak`
% O registo de vacinacao tem de estar associado a um utente que existe
% Conhecimento perfeito negativo
+(-vacinacao_covid(_, IDU, _, _, _)) :: (
    solucoes(IDU, utente(IDU, _, _, _, _, _, _, _, _, _), S),
    comprimento(S, N),
    N == 1
).

% A toma da vacina so pode assumir os valores 1 e 2
+vacinacao_covid(_, _, _, _, T) :: (T > 0, T =< 2).

% A toma da vacina so pode assumir os valores 1 e 2
% Conhecimento perfeito negativo
+(-vacinacao_covid(_, _, _, _, T)) :: (T > 0, T =< 2).

% A segunda toma da vacina so pode ser adminstrada apos a primeira
+vacinacao_covid(_, IDU, _, _, 2) :: (
    solucoes(IDU, vacinacao_covid(_, IDU, _, _, 1), S),
    comprimento(S, N),
    N == 1
).

% A segunda toma da vacina so pode ser adminstrada apos a primeira
% Conhecimento perfeito negativo
+(-vacinacao_covid(_, IDU, _, _, 2)) :: (
    solucoes(IDU, vacinacao_covid(_, IDU, _, _, 1), S),
    comprimento(S, N),
    N == 1
).

% Invariante que impede a insercao de conhecimento perfeito positivo relativo a um registo de vacinacao com vacina interdita (conhecimento imperfeito interdito)
+vacinacao_Covid(STAFF, UTENTE, DATA, VACINA, TOMA) :: (
    solucoes((STAFF, UTENTE, DATA, VACINA, TOMA),
             (vacinacao_Covid(STAFF, UTENTE, DATA, VACINA, TOMA),
              nulo(VACINA)), S),
    comprimento(S, N),
    N == 0
).

\end{lstlisting}

\pagebreak

\subsubsection*{Médico}
O ID, representado por um número inteiro, identifica univocamente um médico e, por isso, não é possível a existência de
mais do que um médico com o mesmo ID. Por outro lado, é imperativo que o médica exerça funções num centro de saúde que,
de facto, exista.

\

\begin{lstlisting}[caption={Invariantes de inserção relativos ao predicado \texttt{medico}}]
% O ID do medico deve ser um numero inteiro e este deve ser unico
+medico(ID, _, _, _, _) :: (
    integer(ID),
    solucoes(ID, medico(ID, _, _, _, _), S),
    comprimento(S, N),
    N == 1
).

% O ID do medico deve ser um numero inteiro e este deve ser unico
% Conhecimento perfeito negativo
+(-medico(ID, _, _, _, _)) :: (
    integer(ID),
    solucoes(ID, medico(ID, _, _, _, _), S),
    comprimento(S, N),
    N == 1
).

% O medico tem de estar associado a um centro de saude que existe
+medico(_, CS, _, _, _) :: (
    solucoes(CS, centro_saude(CS, _, _, _, _), S),
    comprimento(S, N),
    N == 1
).

% O medico tem de estar associado a um centro de saude que existe
% Conhecimento perfeito negativo
+(-medico(_, CS, _, _, _)) :: (
    solucoes(CS, centro_saude(CS, _, _, _, _), S),
    comprimento(S, N),
    N == 1
).

% Invariante que impede a insercao de conhecimento perfeito positivo relativo a um medico com centro de saude interdito (conhecimento imperfeito interdito)
+medico(ID, IDCENTRO, NOME, EMAIL, ESP) :: (
    solucoes((ID, IDCENTRO, NOME, EMAIL, ESP),
              (medico(ID, IDCENTRO, NOME, EMAIL, ESP),
               nulo(IDCENTRO)), S),
    comprimento(S, N),
    N == 0
).

\end{lstlisting}

\subsubsection*{Consulta}
De forma a que o registo de uma consulta seja válido, este deve estar associado a um utente, um médico e a um centro de
saúde que, de facto, existam na base de conhecimento. Por fim, não é admitida a inserção de conhecimento repetido.

\

\begin{lstlisting}[caption={Invariantes de inserção relativos ao predicado \texttt{consulta}}]
% Nao permite a insercao de registos duplicados
+consulta(IDM, IDU, IDC, DATA) :: (
    solucoes((IDM, IDU, IDC, DATA), consulta(IDM, IDU, IDC, DATA), S),
    comprimento(S, N),
    N == 1
).

% Nao permite a insercao de registos duplicados
% Conhecimento perfeito negativo
+(-consulta(IDM, IDU, IDC, DATA)) :: (
    solucoes((IDM, IDU, IDC, DATA), consulta(IDM, IDU, IDC, DATA), S),
    comprimento(S, N),
    N == 1
).

% O registo de uma consulta tem de estar associado a um medico, utente e centro de saude que existam
+consulta(IDM, IDU, CS, _) :: (
    solucoes((IDM, IDU, CS), (medico(IDM, CS, _, _, _),
                              utente(IDU, _, _, _, _, _, _, _, _, CS),
                              centro_saude(CS, _, _, _, _)) , S),
    comprimento(S, N),
    N == 1
).

% O registo de uma consulta tem de estar associado a um medico, utente e centro de saude que existam
% Conhecimento perfeito negativo
+(-consulta(IDM, IDU, CS, _)) :: (
    solucoes((IDM, IDU, CS), (medico(IDM, CS, _, _, _),
                              utente(IDU, _, _, _, _, _, _, _, _, CS),
                              centro_saude(CS, _, _, _, _)) , S),
    comprimento(S, N),
    N == 1
).
`\pagebreak`
% Invariante que impede a insercao de conhecimento perfeito positivo relativo a uma consulta com utente interdito (conhecimento imperfeito interdito)
+consulta(IDM, IDU, IDC, DATA) :: (
    solucoes((IDM, IDU, IDC, DATA),
             (consulta(IDM, IDU, IDC, DATA), nulo(IDU)), S),
    comprimento(S, N),
    N == 0
).
\end{lstlisting}

\subsubsection*{Tratamento}
De forma análoga, o registo de um tratamento deve estar associado a um utente, um elemento do \textit{staff} e a um centro
de saúde que existam efetivamente na base de conhecimento. Neste caso, também não é permitida a inserção de
conhecimento repetido.

\

\begin{lstlisting}[caption={Invariantes de inserção relativos ao predicado \texttt{tratamento}}]
% Nao permite a insercao de registos duplicados
+tratamento(IDS, IDU, IDC, DATA, DESCR) :: (
    solucoes((IDS, IDU, IDC, DATA, DESCR),
             tratamento(IDS, IDU, IDC, DATA, DESCR), S),
    comprimento(S, N),
    N == 1
).

% Nao permite a insercao de registos duplicados
% Conhecimento perfeito negativo
+(-tratamento(IDS, IDU, IDC, DATA, DESCR)) :: (
    solucoes((IDS, IDU, IDC, DATA, DESCR),
             tratamento(IDS, IDU, IDC, DATA, DESCR), S),
    comprimento(S, N),
    N == 1
).

% O registo de um tratamento tem de estar associado a um elemento do staff, utente e centro de saude que existam
+tratamento(IDS, IDU, CS, _, _) :: (
    solucoes((IDS, IDU, CS), (staff(IDS, CS, _, _),
                              utente(IDU, _, _, _, _, _, _, _, _, CS),
                              centro_saude(CS, _, _, _, _)) , S),
    comprimento(S, N),
    N == 1
).
`\pagebreak`
% O registo de um tratamento tem de estar associado a um elemento do staff, utente e centro de saude que existam
% Conhecimento perfeito negativo
+(-tratamento(IDS, IDU, CS, _, _)) :: (
    solucoes((IDS, IDU, CS), (staff(IDS, CS, _, _),
                              utente(IDU, _, _, _, _, _, _, _, _, CS),
                              centro_saude(CS, _, _, _, _)) , S),
    comprimento(S, N),
    N == 1
).

% Invariante que impede a insercao de conhecimento perfeito positivo relativo a um tratamento com centro de saude interdito (conhecimento imperfeito interdito)
+tratamento(IDS, IDU, IDC, DATA, DSCR) :: (
    solucoes((IDS, IDU, IDC, DATA, DSCR),
             (tratamento(IDS, IDU, IDC, DATA, DSCR), nulo(IDC)), S),
    comprimento(S, N),
    N == 0
).

\end{lstlisting}

\subsubsection{Invariantes Associados à Remoção de Conhecimento}
Para a remoção de conhecimento, foi oportuno desenvolver invariantes que, perante toda a lógica do funcionamento do sistema,
não permitissem a remoção de conhecimento relativo a utentes, elementos do \textit{staff}, centros de saúde e médicos
quando estes se encontrassem associados a um determinado registo de vacinação, consulta ou tratamento. Por outro lado, a
remoção destes mesmos registos não é permitida.

\subsubsection*{Utente}
Apenas é possível a remoção de um utente se não existirem registos de vacinação, consulta ou tratamento, restrição essa
imposta pelos seguintes invariantes.

\

\begin{lstlisting}[caption={Invariantes de remoção relativos ao predicado \texttt{utente}}]
% Nao permite a remocao de um utente se existirem registos de vacinacao a si associados
-utente(ID, _, _, _, _, _, _, _, _, _) :: (
    solucoes(ID, vacinacao_Covid(_, ID, _, _, _), S),
    comprimento(S, N),
    N == 0
).

% Nao permite a remocao de um utente se existirem registos de consultas a si associaos
-utente(ID, _, _, _, _, _, _, _, _, _) :: (
    solucoes(ID, consulta(_, ID, _, _), S),
    comprimento(S, N),
    N == 0
).

% Nao permite a remocao de um utente se existirem
-utente(ID, _, _, _, _, _, _, _, _, _) :: (
    solucoes(ID, tratamento(_, ID, _, _, _), S),
    comprimento(S, N),
    N == 0
).
\end{lstlisting}

\subsubsection*{\textit{Staff}}
Foram construídos os seguintes invariantes no sentido de não permitir a remoção de um elemento do \textit{staff}
caso existam registos de vacinação ou de tratamento a si associados.

\

\begin{lstlisting}[caption={Invariantes de remoção relativos ao predicado \texttt{staff}}]
% Nao permite a remocao de um elemento do staff se existirem registos de vacinacao a si associaos
-staff(ID, _, _, _) :: (
    solucoes(ID, vacinacao_Covid(ID, _, _, _, _), S),
    comprimento(S, N),
    N == 0
).

% Nao permite a remocao de um elemento do staff se existirem registos de tratamentos a si associaos
-staff(ID, _, _, _) :: (
    solucoes(ID, tratamento(ID, _, _, _, _), S),
    comprimento(S, N),
    N == 0
).
\end{lstlisting}

\subsubsection*{Centro de Saúde}
Para ser possível a remoção de um centro de saúde, é necessário que não existam utentes, médicos ou elementos do \textit{staff}
a si associados, restrição essa que é imposta pelos seguintes invariantes.

\

\begin{lstlisting}[caption={Invariantes de remoção relativos ao predicado \texttt{centro\_saude}}]
% Nao permite a remocao de um centro de saude se existirem utentes a si associaos
-centro_saude(CS, _, _, _, _) :: (
    solucoes(CS, utente(_, _, _, _, _, _, _, _, _, CS), S),
    comprimento(S, N),
    N == 0
).
`\pagebreak`
% Nao permite a remocao de um centro de saude se existirem elementos do staff a si associaos
-centro_saude(CS, _, _, _, _) :: (
    solucoes(CS, staff(_, CS, _, _), S),
    comprimento(S, N),
    N == 0
).

% Nao permite a remocao de um centro de saude se existirem medicos a si associaos
-centro_saude(CS, _, _, _, _) :: (
    solucoes(CS, medico(_, CS, _, _, _), S),
    comprimento(S, N),
    N == 0
).
\end{lstlisting}

\subsubsection*{Vacinação}
O seguinte invariante garante que não é possível remover registos de vacinação.

\begin{lstlisting}[caption={Invariante de remoção relativos ao predicado \texttt{vacinacao\_covid}}]
% Nao permite a remocao de registos de atos de vacinacao
-vacinacao_Covid(_, _, _, _, _) :: fail.
\end{lstlisting}

\subsubsection*{Médico}
Para ser possível a remoção de um médico, é necessário que não existam registos de consultas a si associados, restrição
essa que é imposta pelo seguinte invariante.

\

\begin{lstlisting}[caption={Invariante de remoção relativo ao predicado \texttt{medico}}]
% Nao permite a remocao de um medico se existirem consultas a si associadas
-medico(ID, _, _, _, _) :: (solucoes(ID, consulta(ID, _, _, _), S),
                            comprimento(S, N),
                            N == 0).
\end{lstlisting}

\subsubsection*{Consulta}
O seguinte invariante garante que não é possível remover registos de consultas.
\
\begin{lstlisting}[caption={Invariante de remoção relativo ao predicado \texttt{consulta}}]
% Nao permite a remocao de registos de consultas
-consulta(_, _, _, _) :: fail.
\end{lstlisting}
\subsubsection*{Tratamento}
O seguinte invariante garante que não é possível remover registos de tratamentos.
\
\begin{lstlisting}[caption={Invariante de remoção relativo ao predicado \texttt{tratamento}}]
% Nao permite a remocao de registos de tratamentos
-tratamento(_, _, _, _, _) :: fail.
\end{lstlisting}